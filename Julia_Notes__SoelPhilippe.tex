% Options for packages loaded elsewhere
\PassOptionsToPackage{unicode}{hyperref}
\PassOptionsToPackage{hyphens}{url}
%
\documentclass[
]{article}
\usepackage{amsmath,amssymb}
\usepackage{iftex}
\ifPDFTeX
  \usepackage[T1]{fontenc}
  \usepackage[utf8]{inputenc}
  \usepackage{textcomp} % provide euro and other symbols
\else % if luatex or xetex
  \usepackage{unicode-math} % this also loads fontspec
  \defaultfontfeatures{Scale=MatchLowercase}
  \defaultfontfeatures[\rmfamily]{Ligatures=TeX,Scale=1}
\fi
\usepackage{lmodern}
\ifPDFTeX\else
  % xetex/luatex font selection
\fi
% Use upquote if available, for straight quotes in verbatim environments
\IfFileExists{upquote.sty}{\usepackage{upquote}}{}
\IfFileExists{microtype.sty}{% use microtype if available
  \usepackage[]{microtype}
  \UseMicrotypeSet[protrusion]{basicmath} % disable protrusion for tt fonts
}{}
\makeatletter
\@ifundefined{KOMAClassName}{% if non-KOMA class
  \IfFileExists{parskip.sty}{%
    \usepackage{parskip}
  }{% else
    \setlength{\parindent}{0pt}
    \setlength{\parskip}{6pt plus 2pt minus 1pt}}
}{% if KOMA class
  \KOMAoptions{parskip=half}}
\makeatother
\usepackage{xcolor}
\usepackage{longtable,booktabs,array}
\usepackage{calc} % for calculating minipage widths
% Correct order of tables after \paragraph or \subparagraph
\usepackage{etoolbox}
\makeatletter
\patchcmd\longtable{\par}{\if@noskipsec\mbox{}\fi\par}{}{}
\makeatother
% Allow footnotes in longtable head/foot
\IfFileExists{footnotehyper.sty}{\usepackage{footnotehyper}}{\usepackage{footnote}}
\makesavenoteenv{longtable}
\setlength{\emergencystretch}{3em} % prevent overfull lines
\providecommand{\tightlist}{%
  \setlength{\itemsep}{0pt}\setlength{\parskip}{0pt}}
\setcounter{secnumdepth}{-\maxdimen} % remove section numbering
\ifLuaTeX
  \usepackage{selnolig}  % disable illegal ligatures
\fi
\IfFileExists{bookmark.sty}{\usepackage{bookmark}}{\usepackage{hyperref}}
\IfFileExists{xurl.sty}{\usepackage{xurl}}{} % add URL line breaks if available
\urlstyle{same}
\hypersetup{
  hidelinks,
  pdfcreator={LaTeX via pandoc}}
\title{Personal Notes on the \textbf{Julia} Programming language}
\author{\texttt{Soel Philippe}}
\date{July 2023}
\begin{document}
\maketitle
\tableofcontents
\newpage
\begin{minipage}[c][7cm][t]{\textwidth}
\vspace*{\fill}
\centering
{\Huge \textbf{WARNING}}\\
\textbf{These notes are still being written and therefore contain typos (be kind to report them).\\
These notes are not, in any manner, guaranteed to be accurate. Any personal opinion therein is my
sole mistake, do not blindly relate to these notes.\\
Refer only to the official documentation (https://docs.julialang.org/en/v1/)}.
\vspace*{\fill}
\end{minipage}
\newpage
\hypertarget{introduction}{%
\section{Introduction}\label{introduction}}
The \textbf{Julia} programming language aims to fill the role of a
programming language eliminating the performance trade-off, providing a
single environment productive enough for prototyping and efficient
enough for deploying performance-intensive applications. The
\textbf{Julia} aims to be a flexible dynamic language, appropriate for
scientific and numerical computing, with performance comparable to
traditional statically-typed languages.\\
Julia's compiler is different from the interpreters used for languages
like \textbf{Python} or \textbf{R} in such a way that one may find it's
performance unintuitive at first but one is fully acknowledged of how
\textbf{Julia} works, it's easy to write code that's nearly as fast as
\textbf{C}.

\textbf{Julia} implements \emph{type inference}, \emph{just-in-time
(JIT) compilation} and \emph{ahead-of-time compilation} implemented
using \emph{LLVM}. It is multi-paradigm, combining features of
imperative, functional and object oriented programming.\\
To achieve its flexibility, \textbf{Julia} builds upon the lineage of
mathematical programming languages but also borrows much from popular
dynamic languages, including \textbf{Lisp}, \textbf{Perl},
\textbf{Python}, \textbf{Lua}, and \textbf{Ruby}.

\hypertarget{julia-departures-from-typical-pl}{%
\subsection{Julia Departures from Typical
PL:}\label{julia-departures-from-typical-pl}}

\textbf{Julia} \emph{Base} and the standard library are written in
\textbf{Julia} itself.\\
\textbf{Julia} has a rich language of types for constructing and
describing objects that can also optionally be used to make type
declarations.\\
In \textbf{Julia}, every object has a type. The lack of type
declarations in most dynamic languages, however, means that one cannot
instruct the compiler about the types of values, and often cannot
explicitly talk about types at all. However, in static languages while
one can (and usually must) annotate types for the compiler, types exist
only at compile time and cannot be manipulated or expressed at run
time.\\
In \textbf{Julia}, types are themselves run-time objects and can also be
used to convey information to the compiler.\\
\emph{Types} and \emph{Multiple Dispatch} are the core unifying features
of \textbf{Julia}.\\
In \textbf{Julia}, operators are functions with special notation.

\hypertarget{summary-of-julia-features}{%
\subsection{Summary of Julia
features:}\label{summary-of-julia-features}}

\begin{itemize}
\tightlist
\item
  Free and Open Source (MIT License).
\item
  User-defined types are as fast and compact as built-ins.
\item
  No need to vectorize code for perfomance.
\item
  Designed for Parallelism and distributed computation.
\item
  Lightweight ``green'' threading (\emph{coroutines}).
\item
  Efficient support for \emph{Unicode} characters.
\item
  Call \textbf{C} functions directly (no wrappers or APIs needed).
\item
  Shell-like capabilities for managing other processes.
\item
  \textbf{Lisp}-like macros and other metaprogramming facilities.
\end{itemize}

\newpage

\hypertarget{getting-started}{%
\section{Getting Started}\label{getting-started}}

\textbf{Julia} installation is straightforward whether using precompiled
binaries or compiling from source. It can be downloaded and installed by
following \href{https://julialang.org/downloads/}{Julia-lang download
and Installation}.

The easiest way to learn and experiment with \textbf{Julia} is by
starting an interactive session known as \texttt{REPL} through running
\texttt{julia} from the command line.

Exit the interactive session by pressing \texttt{CTRL-D} or typing
\texttt{exit()}.\\
To evaluate an expression written in a source file \emph{file.jl}, write
\texttt{include("file.jl")}.\\
To run a code in a file non-interactively run it as:

\begin{verbatim}
$ julia script file.jl
\end{verbatim}

Get help by typing \texttt{?} into an empty \textbf{Julia} REPL prompt.

\newpage

\hypertarget{variables}{%
\section{Variables}\label{variables}}

\textbf{Julia} provides an extremely flexible system for naming
variables. Variable names are case-sensitive, and have no semantic
meaning. In \textbf{Julia} \texttt{REPL} and several other julia editing
environments, one can type many Unicode math symbols by typing the
backslashed \LaTeX symbol name followed by \texttt{tab}. For example,
the variable name \(\delta\) can be entered by typing
\texttt{\textbackslash{}delta-tab}. \textbf{Julia} will let you redefine
built-in constants and functions if needed (though not recommended).

\hypertarget{allowed-variable-names}{%
\subsection{Allowed Variable Names}\label{allowed-variable-names}}

Variable names must begin with a letter (A-Z or a-z), underscore, or a
subset of Unicode code points greater than 00A0. In particular, Unicode
character categories Lu/Ll/Lt/Lm/Lo/Nl (letters), Sc/So (currency and
other symbols), and a few other letter-like characters are allowed.

Subsequent characters may include \texttt{!} and digits and other
characters in categories Nd/No as well as other Unicode code points:
diacrits and other modifying marks (categories Mn/Mc/Me/Sk), some
punctuation connectors (category Pc), primes, and a few other
characters.

Operators like \texttt{+} are also valid identifiers, but are parsed
specially. In some contexts, operators can be used just like variables;
for example (\texttt{*}) refers to multiplication or string
concatenation and \texttt{*\ =\ h} will reassign it.

A particular class of variable names is one that contains only
underscores. These identifiers can only be \emph{left values}.

\hypertarget{assignment-expressions-and-assignment-versus-mutation}{%
\subsection{Assignment expressions and assignment versus
mutation}\label{assignment-expressions-and-assignment-versus-mutation}}

An assignment \texttt{variable\ =\ value} binds the name
\texttt{variable} to the \texttt{value} computed on the right-hand suden
abd rge whole assignment is treated by \textbf{Julia} as an expression
equal to the right-hand-side value. Therefore, assignments can be
chained or used in other expressions. That is the reason why their
result is shown in the \texttt{REPL} as the value of the right-hand
side.

A common confusion is the distinction between \textbf{assignment}
\emph{which is giving a new name to a value} while \textbf{mutation}
\emph{changing a value}.

\hypertarget{stylistic-conventions}{%
\subsection{Stylistic Conventions}\label{stylistic-conventions}}

\begin{itemize}
\tightlist
\item
  Names of variables are in lower case.
\item
  Word separation can be indicated by underscores ('\_'), but
  discouraged.
\item
  Names of Types and Modules befin with capital letter and word-sep is
  on CamelCase.
\item
  Names of functions and macros are in lower case, without Underscores.
\item
  Functions that write to their arguments have names ending with
  \texttt{!}.
\end{itemize}

\newpage

\hypertarget{integers-and-floating-point-numbers}{%
\section{Integers and Floating-Point
Numbers}\label{integers-and-floating-point-numbers}}

\begin{longtable}[]{@{}lcccc@{}}
\toprule\noalign{}
Type & Signed & Number of Bits & Smallest Value & Largest Value \\
\midrule\noalign{}
\endhead
\bottomrule\noalign{}
\endlastfoot
Int8 & yes & 8 & \(-2^{7}\) & \(2^{7} - 1\) \\
UInt8 & no & 8 & \(0\) & \(2^{8} - 1\) \\
Int16 & yes & 16 & \(-2^{15}\) & \(2^{15} - 1\) \\
UInt16 & no & 16 & \(0\) & \(2^{16} - 1\) \\
Int32 & yes & 32 & \(-2^{31}\) & \(2^{31} - 1\) \\
UInt32 & no & 32 & \(0\) & \(2^{32}-1\) \\
Int64 & yes & 64 & \(-2^{63}\) & \(2^{63} - 1\) \\
UInt64 & no & 64 & \(0\) & \(2^{64} - 1\) \\
Int128 & yes & 128 & \(-2^{127}\) & \(2^{127} - 1\) \\
UInt128 & yes & 128 & \(0\) & \(2^{128} - 1\) \\
Bool & NA & 8 & false(0) & true(1) \\
\end{longtable}

Additionally, full support for \textbf{Complex and Rational Numbers} is
built on top of these primitive numeric types. All numeric types
interoperate naturally without explicit casting thanks to the
\textbf{Julia}'s user-extensible type promotion system.

\begin{longtable}[]{@{}lcc@{}}
\toprule\noalign{}
Type & Signed & Number of Bits \\
\midrule\noalign{}
\endhead
\bottomrule\noalign{}
\endlastfoot
Float16 & half & 16 \\
Float32 & single & 32 \\
Float64 & double & 64 \\
\end{longtable}

\hypertarget{integer-types}{%
\subsection{Integer Types}\label{integer-types}}

The \textbf{Julia} internal variable \texttt{Sys.WORD\_SIZE} indicates
whether the target system is 32-bit or 64-bit.

In \textbf{Julia} exceeding the maximum representable value of a given
type results in a wraparound behavior. To know about the maximum
representable value of a given type \texttt{T}, type
\texttt{typemax(T)}.

\textbf{Julia} has the \texttt{BigInt} type in \emph{Arbitrary Precision
Arithmetic} is recommended instead; implemented for example as
\texttt{big(10)\^{}19}.

\hypertarget{floating-point-numbers}{%
\subsection{Floating-Point Numbers}\label{floating-point-numbers}}

Floating-point numbers have two eros, positive zero and negative zero.
They are equal to each other but have different binary representations,
as can be seen using \texttt{bitstring} function. Check it with the
commands \texttt{bitstring(0.0)} and \texttt{bitstring(-0.0)}. There are
three specified standard floating-pooint values that do not correspond
to any point on the real number line: \texttt{-Inf}, \texttt{Inf} and
\texttt{NaN}.

\hypertarget{machine-epsilon}{%
\subsection{Machine epsilon}\label{machine-epsilon}}

Most real numbers cannot be represented exactly with floating-point
numbers, and so for many purposes it is important to know the distance
between two adjacent representable floatin-point numbers, which is often
known as \emph{machine epsilon}. \textbf{Julia} provides \texttt{eps}
which gives the distance between 1.0 and the next larger representable
flaoting-point value. Try it as \texttt{eps(Float64)}. (see also
\texttt{nexfloat}, \texttt{prevfloat})

Floating-point arithmetic entails many subtleties which can be
surprising to users who are unfamiliar with the low-level implementation
details. However, these subtleties are described in detail in most books
on scientific computation, and also in the following references:

\begin{itemize}
\tightlist
\item
  The definitive guide to floating point arithmetic is the IEEE 754-2008
  Standard; however, it is not available for free online.
\item
  For a brief but lucid presentation of how floating-point numbers are
  represented, see John D. Cook's article on the subject as well as his
  introduction to some of the issues arising from how this
  representation differs in behavior from the idealized abstraction of
  real numbers.
\item
  Also recommended is Bruce Dawson's series of blog posts on
  floating-point numbers.
\item
  For an excellent, in-depth discussion of floating-point numbers and
  issues of numerical accuracy encountered when computing with them, see
  David Goldberg's paper What Every Computer Scientist Should Know About
  Floating-Point Arithmetic.
\item
  For even more extensive documentation of the history of, rationale
  for, and issues with floatingpoint numbers, as well as discussion of
  many other topics in numerical computing, see the collected writings
  of William Kahan, commonly known as the ``Father of Floating-Point''.
  Of particular interest may be An Interview with the Old Man of
  Floating-Point.
\end{itemize}

\hypertarget{arbitrary-precisiion-arithmetic}{%
\subsection{Arbitrary Precisiion
Arithmetic}\label{arbitrary-precisiion-arithmetic}}

To allow computations with arbitrary-precision integers and floating
point numbers, Julia wraps the GNU Multiple Precision Arithmetic Library
(GMP) and the GNU MPFR Library, respectively. The \texttt{BigInt} and
\texttt{BigFloat} types are available in \textbf{Julia} for arbitrary
precision integer and floating point numbers respectively.

Constructors exist to create these types from primitive numerical types,
and the string literal \texttt{@big\_str} or \texttt{parse} can be used
to construct them from \texttt{AbstractString}s. \texttt{BigInt}s can
also be input as integer literals when they are too big for other
built-in integer types. Note that as there is no unsigned
arbitrary-precision integer type in Base (\texttt{BigInt} is sufficient
in most cases), hexadecimal, octal and binary literals can be used (in
addition to decimal literals).

The default precision (in number of bits of the significand) and
rounding mode of \texttt{BigFloat} operations can be changed globally by
calling \texttt{setprecision} and \texttt{setrounding}, and all further
calculations will take these changes in account. Alternatively, the
precision or the rounding can be changed only within the execution of a
particular block of code by using the same functions with a \texttt{do}
block:

\begin{verbatim}
setrounding(BigFloat, RoundUp) do
  BigFloat(1) + parse(BigFloat, "0.1")
end
\end{verbatim}

\begin{verbatim}
setrounding(BigFloat, RoundDown) do
  BigFloat(1) + parse(BigFloat, "0.1")
end
\end{verbatim}

\begin{verbatim}
setprecision(40) do
  BigFloat(1) + parse(BigFloat, "0.1")
end
\end{verbatim}

\hypertarget{numeric-literal-coefficients}{%
\subsection{Numeric Literal
Coefficients}\label{numeric-literal-coefficients}}

To make common numeric formulae and expressions clearer, Julia allows
variables to be immediately preceded by a numeric literal, implying
multiplication:

\begin{verbatim}
julia> z = 1.023
1.023

julia> 2z^7 - 3z^2 + 5z - 12
-7.679497448724284
\end{verbatim}

The precedence of numeric literal coefficients is slightly lower than
that of unary operators such as negation. So -2x is parsed as (-2) * x.
The precedence of numeric literal coefficients used for implicit
multiplication is higher than other binary operators such as
multiplication (*), and division (/, , and //). Neither juxtaposition of
two parenthesized expressions, nor placing a variable before a
parenthesized expression, however, can be used to imply multiplication.

\newpage

\hypertarget{mathematical-operations-and-elementary-functions}{%
\section{Mathematical Operations and Elementary
Functions}\label{mathematical-operations-and-elementary-functions}}

\hypertarget{arithmetic-operators}{%
\subsection{Arithmetic Operators}\label{arithmetic-operators}}

\begin{longtable}[]{@{}lcc@{}}
\toprule\noalign{}
Expression & Name & Description \\
\midrule\noalign{}
\endhead
\bottomrule\noalign{}
\endlastfoot
+x & unary plus & the identity operation \\
-x & unary minus & maps values to their additive inverses \\
x + y & binary plus & performs addition \\
x * y & times & performs multiplication \\
x / y & divide & performs division \\
x ÷ y & x & x/y, truncated to an integer \\
x ~y & inverse divide & equivalent to y / x \\
x \^{} y & power & raises x to yth power \\
x \% y & remainder & equivalent to \texttt{rem(x,\ y)} \\
\end{longtable}

A numeric literal placed directly before an identifier or parentheses,
e.g.~2x or 2(x+y), is treated as a multiplication, except with higher
precedence than other binary operations.

\hypertarget{boolean-operators}{%
\subsection{Boolean Operators}\label{boolean-operators}}

\begin{longtable}[]{@{}lc@{}}
\toprule\noalign{}
Expression & Name \\
\midrule\noalign{}
\endhead
\bottomrule\noalign{}
\endlastfoot
!x & negation \\
x \&\& y & short-circuit and \\
x \textbar\textbar{} y & short-circuit or \\
\end{longtable}

\hypertarget{vectorized-dot-operators}{%
\subsection{Vectorized ``dot''
Operators}\label{vectorized-dot-operators}}

For every binary operation, there is a corresponding ``dot'' operation
that is automatically defined to perform operators element-by-element on
arrays. For eg. \texttt{{[}1.25,\ 4.0,\ 7,\ 5.39{]}\ .\^{}\ 2.5} is
equivalent to
\texttt{{[}1.25\^{}(2.5),\ 4.0\^{}(2.5),\ 7\^{}(2.5),\ 5.39\^{}(2.5){]}}.

More specifically, \texttt{a\ .\^{}\ b} is parsed as the ``dot'' call
\texttt{(\^{}).(a,\ b)}, which performs a broadcast operation it can
combine arrays and scalars, arrays of the same size and even arrays of
different shapes.

In \textbf{Julia} there is a macro that fuses the ``dots'' operations.
Eg. if \texttt{A} is a multi-dimensional array
\texttt{@.\ 2A\^{}2\ +\ exp(cos(A))} is equivalent to applying the
function \texttt{f(x)\ =\ 2x\^{}2\ +\ exp(cos(x))} to \texttt{A}
element-wise.

\hypertarget{numeric-comparisons}{%
\subsection{Numeric Comparisons}\label{numeric-comparisons}}

\begin{longtable}[]{@{}lc@{}}
\toprule\noalign{}
Operator & Name \\
\midrule\noalign{}
\endhead
\bottomrule\noalign{}
\endlastfoot
== & equality \\
!= & inequality \\
\textless{} & less than \\
\textgreater{} & greater than \\
\textless= & less than or equal to \\
\textgreater= & greater than or equal to \\
\end{longtable}

\textbf{Julia} provides additional functions to test numbers for special
values, which can be useful in situations like hash key comparisons such
as: \texttt{isequal}, \texttt{isfinite}, \texttt{isinf},
\texttt{isnan}\ldots{}

In \textbf{Julia}, unlike most languages, comparisons can be arbitrarily
chained:
\texttt{1\ \textless{}\ 2\ \textless{}=\ 2\ \textless{}\ 3\ ==\ 3\ \textgreater{}\ 2\ \textgreater{}=\ 1\ ==\ 1\ 1\ \textless{}\ 3\ !=\ 5}

\hypertarget{rounding-division-and-sign-functions}{%
\subsection{Rounding, Division, and Sign
Functions}\label{rounding-division-and-sign-functions}}

\begin{longtable}[]{@{}lcc@{}}
\toprule\noalign{}
Function & Description & Return Type \\
\midrule\noalign{}
\endhead
\bottomrule\noalign{}
\endlastfoot
round(x) & round x to the nearest integer & typeof(x) \\
round(T, x) & round x to the nearest integer & T \\
floor(x) & round x towards -Inf & typeof(x) \\
floor(T, x) & round x towards -Inf & T \\
ceil(x) & round x towards +Inf & typeof(x) \\
ceil(T, x) & round x towards +Inf & T \\
trunc(x) & round x towards zero & typeof(x) \\
trunc(T, x) & round x towards zero & T \\
\end{longtable}

\begin{longtable}[]{@{}
  >{\raggedright\arraybackslash}p{(\columnwidth - 2\tabcolsep) * \real{0.4444}}
  >{\centering\arraybackslash}p{(\columnwidth - 2\tabcolsep) * \real{0.5556}}@{}}
\toprule\noalign{}
\begin{minipage}[b]{\linewidth}\raggedright
Function
\end{minipage} & \begin{minipage}[b]{\linewidth}\centering
Description
\end{minipage} \\
\midrule\noalign{}
\endhead
\bottomrule\noalign{}
\endlastfoot
div(x, y) & truncated division; quotient rounded towards zero \\
fld(x, y) & floored division; quotient rounded towards -Inf \\
cld(x, y) & ceiling division; quotient rounded towards +Inf \\
rm(x, y) & remainder; satisfies x == div(x, y)*y + rem(x, y); sign
matches x \\
mod(x, y) & modulus; satisfies x == fld(x, y)*y + mod(x, y); sign
matches y \\
mod1(x, y) & mod with offset 1; returns \(r\in (0, y]\) for y
\textgreater{} 0 or \(r\in [y, 0)\) for y \textless{} 0, where mod(r, y)
== mod(x, y) \\
mod2pi(x) & modulus with respect to 2pi; 0 \textless= mod2pi(x)
\textless{} 2pi \\
divrem(x, y) & returns (div(x, y), rem(x, y)) \\
fldmod(x, y) & returns (fld(x,y), mod(x,y)) \\
gcd(x, y\ldots) & greatest positive common divisor of x, y, \ldots{} \\
abs(x) & a positive value with the magnitude of x \\
abs2(x) & the square magnitude of x \\
sign(x) & indicates the sign of x, returning -1, 0, or +1 \\
signbit(x) & indicates whether the sign bit is on (true) or off
(false) \\
copysign(x, y) & a value with the magnitude of x and the sign of y \\
flipsign(x, y) & a value with the magnitude of x and the sign of x*y \\
sqrt(x) & square root of x \\
cbrt(x) & cube root of x \\
hypot(x, y) & hypotenuse of right-angled triangle with other sides of
length x and y \\
exp(x) & natural exponetial function at x \\
expm1(x) & accurate exp(x)-1 for x near zero \\
ldexp(x, n) & x*2\^{}n computed efficiently for integer values of n \\
log(x) & natural logarithm of x \\
log2(x) & base b logarithm of x \\
log10(x) & base 10 logarithm of x \\
log1p(x) & accurate log(1+x) for x near zero \\
exponent(x) & binary exponent of x \\
significand(x) & binary significand (a.k.a mantissa) of a floating-point
number x \\
\end{longtable}

Trigonometric functions are implemented into \textbf{Julia}
(\texttt{cos}, \texttt{cosh}, \texttt{acosh}\ldots) as well as
trigonometric functions in degrees instead of radians (\texttt{cosd},
\texttt{acosd}\ldots).

\newpage

\hypertarget{complex-and-rational-numbers}{%
\section{Complex and Rational
Numbers}\label{complex-and-rational-numbers}}

\textbf{Julia} includes predefined types for both complex and rational
numbers, and supports all the standard Mathematical Operations and
Elementary Functions on them. Conversion and Promotion are defined so
that operations on any combinatiion of predefined numeric types, whether
primitive or composite, behave as expected.

\hypertarget{complex-numbers}{%
\subsection{Complex Numbers}\label{complex-numbers}}

The global constant \texttt{im} is bound to the complex number \(i\),
representing what's commonly known as the square root of \(-1\). All
standard arithmetic operations can be performed with complex numbers.

\begin{verbatim}
julia> (.3+7im)^1.3
-5.072305883244214 + 11.495153867536395im

julia> (-1 + 2im)^(1 + 1im)
-0.27910381075826657 + 0.08708053414102428im
\end{verbatim}

Some usual functions related to complex numbers manipulation are
implemented:

\begin{longtable}[]{@{}lc@{}}
\toprule\noalign{}
Function & Description \\
\midrule\noalign{}
\endhead
\bottomrule\noalign{}
\endlastfoot
imag(z) & imaginary part of z \\
conj(z) & complex conjugate of z \\
abs(z) & absolute value of z \\
abs2(z) & square absolute value of z \\
angle(z) & phase angle in radians \\
\end{longtable}

\ldots{} however calling \texttt{sqrt(-1)} doesn't work. One should call
it as \texttt{sqrt(-1\ +\ 0im)}.

\begin{verbatim}
julia> sqrt(-1)
ERROR: DomainError with -1.0:
sqrt will only return a complex result if called with a complex argument. Try sqrt(Complex(x)).
Stacktrace:
 [1] throw_complex_domainerror(f::Symbol, x::Float64)
   @ Base.Math ./math.jl:33
 [2] sqrt
   @ ./math.jl:677 [inlined]
 [3] sqrt(x::Int64)
   @ Base.Math ./math.jl:1491
 [4] top-level scope
   @ REPL[5]:1
\end{verbatim}

\hypertarget{rational-numbers}{%
\subsection{Rational Numbers}\label{rational-numbers}}

\textbf{Julia} has a rational number type to represent exact ratios of
integers. RAtionals are constructed using the operator \texttt{//}.

\begin{verbatim}
julia> -4//12
-1//3
\end{verbatim}

This normalized form for a ratio of integers us unique, so equality of
rational values can be tested by checking for equality of the numerator
and denominator. The standardized numerator and denominator of a
rational number can be extracted using the built-in \texttt{numerator}
and \texttt{denominator} functions.

\newpage

\hypertarget{strings}{%
\section{Strings}\label{strings}}

\textbf{Julia} supports Unicode so that dealing with plain ASCII text is
as simple and efficient as possible. However, there are a few noteworthy
high-level features about \textbf{Julia}'s strings:

\begin{itemize}
\tightlist
\item
  The buit-in concrete type used for strings (and string literals) in
  \textbf{Julia} is \emph{String}. This supports the full range of
  Unicode characters via the UTF-8 encoding. (A \texttt{transcode}
  function is provided to convert to/from Unicode encodings.
\item
  All string types are subtypes of the abstract type
  \emph{AbstractString}, and external packages define additional
  *AbstractString** subtypes (e.g.~for other encodings). If you deifne a
  function expecting a string argument, you should declare the type as
  \emph{AbstractString} in order to accept any string type.
\item
  Like \textbf{C} and \textbf{Java}, but unlike most dynamic languauges,
  \textbf{Julia} has a first-class type for representing a single
  character, called \emph{AbstractChar}. The built-in \emph{Char}
  subtype of \emph{AbstractChar} is a 32-bit primitive type that can
  represent any Unicode character (and which is based on the UTF-8
  encoding).
\item
  As in \textbf{Java}, strings are immutable: the value of an
  \emph{AbstractString} object cannot be changed. To construct a
  different string value, you construct a new string from parts of other
  strings.
\item
  Conceptually, a string is a partial function from indices to
  characters: for some index values, no character value is returned, and
  instread an exception is thrown. This allows for efficient indexing
  into strings by the byte index of an encoded representation rather
  than by a character index, which cannot be implemented both
  efficiently and simply for variable-width encodings of Unicode
  strings.
\end{itemize}

\hypertarget{characters}{%
\subsection{Characters}\label{characters}}

A Char value represents a single character, it is just a 32-bit
primitive type with a special literal representation and appropriate
arithmetic behaviors, and which can be converted to a numeric value
representing a Unicode code point. \textbf{Julia} packages may define
other subtypes of AbstractChar, e.g.~to optimize operations for other
text encodings.

One can easily convert a \emph{Char} to its integer value with
\texttt{c\ =\ Int(\textquotesingle{}x\textquotesingle{})} as well as
convert an integer value back to a \emph{Char} just as easily as
\texttt{Char(120)}.

Not all integer values are valid Unicode code points, but for
performance, the Char conversion does not check that every character
value is valid. If you want to check that each converted value is a
valid code point, use the \texttt{isvalid} function
\texttt{isvalid(Char,\ 0x110000)}.

Comparisons and a limited amount of arithmetic can be performed with
\emph{Char}.

\begin{verbatim}
julia> 'A' < 'a'
true

julia> 'A' <= 'a' <= 'Z'
false

julia> 'x' - 'a'
23

julia 'A' + 3
'D': ASCII/Unicode U+0044 (category Lu: Letter, uppercase)
\end{verbatim}

\hypertarget{string-basics}{%
\subsection{String Basics}\label{string-basics}}

String literals are delimited by doube quotes or triple double quotes.
Log lines in strings can be broken up by preceding the newline with a
backslash ().

To extract a character from a string \texttt{str}, one can index into it
as \texttt{str{[}begin{]}} for the first character, \texttt{str{[}i{]}}
for the \texttt{i}-th index and \texttt{str{[}end{]}} for the last
character. Recall that in \textbf{Julia} indexes start from 1.

Many Julia objects, including strings, can be indexed with integers. The
index of the first element (the first character of a string) is returned
by \texttt{firstindex(str)}, and the index of the last element
(character) with \texttt{lastindex(str)}. The keywords \texttt{begin}
and end can be used inside an indexing operation as shorthand for the
first and last indices, respectively, along the given dimension. String
indexing, like most indexing in \textbf{Julia}, is 1-based:
\texttt{firstindex} always returns 1 for any \emph{AbstractString}. As
we will see below, however, \texttt{lastindex(str)} is not in general
the same as \texttt{length(str)} for a string, because some Unicode
characters can occupy multiple ``code units''.

\hypertarget{concatenation}{%
\subsection{Concatenation}\label{concatenation}}

Concatenation of \texttt{str1}, \texttt{str2}, \ldots, \texttt{strn} can
be performed using the function \texttt{string} as
\texttt{string(str1,\ str2,\ ...,\ strn)}.\\
\textbf{Julia} also provides \texttt{*} for string concatenation as
\texttt{str1\ *\ str2\ *\ str3}.

\hypertarget{interpolation}{%
\subsection{Interpolation}\label{interpolation}}

Constructing strings using concatenation can become a bit cumbersome,
however. To reduce the need for these verbose calls to \texttt{string}
or repeated ``multiplications'', \textbf{Julia} allows interpolation
into string literals using \texttt{\$}, as in Perl.

\begin{verbatim}
julia> s1 = "Molly";

julia> s2 = "Percocet";

julia> s = "$s1, $s2"
"Molly, Percocet"

julia> "47 + 53 = $(47 + 53)"
"47 + 53 = 100"
\end{verbatim}

\hypertarget{triple-quoted-string-literals}{%
\subsection{Triple-Quoted String
Literals}\label{triple-quoted-string-literals}}

When strings are created using triple-quotes (``\,``\,'' \ldots{}
``\,``\,``) they have some special behavior that can be useful for
creating longer blocks of text. First, triple-quoted strings are also
dedented to the level of the least-indented line. This is useful for
defining strings within code that is indented.

Triple-quoted string literals can contain " characters without escaping.
Note that line breaks in literal strings, whether single or
triple-quoted, result in a newline (LF) character
\texttt{\textbackslash{}n} in the string, even if your editor uses a
carriage return \texttt{\textbackslash{}r} (CR) or (CRLF) combination to
end lines. To include a CR in a string use an explicit escapte
\texttt{\textbackslash{}r}; for example, you can enter the literal
string ``a CRLF line ending
\texttt{\textbackslash{}r\textbackslash{}n}''.

\hypertarget{common-operations}{%
\subsection{Common Operations}\label{common-operations}}

Strings are lexicographically ordered.\\
One can search for the index of a particular character using the
\texttt{findfirst} and \texttt{findlast} functions. These functions
allow to start the search for a character at a given offset by using the
functions \texttt{findnext} and \texttt{findprev} with three
arguments.\\
One can use the function \texttt{occursin} to check if a function is
found within a string.

\hypertarget{regular-expressions}{%
\subsection{Regular Expressions}\label{regular-expressions}}

\textbf{Julia} uses version 2 of Perl-compatible regular expressions
(regexes), as provided by the PCRE library (see PCRE2 syntax description
for mode details). Regular expressions are related to strings in two
ways the obvious connection is that regular expressions are used to find
regular patterns in strings; the other connection is that regular
expressions are themselves input as strings, which are parsed into a
state machine that can be used to efficiently search for patterns in
strings. In \textbf{Julia}, regular expressions are input using
non-standard string literals prefixed with various identifiers beginning
with \texttt{r}. The most basic regular expression literal without any
options turned on just uses \texttt{r"..."}.

\begin{verbatim}
julia> re = r"^\s*(?:#|$)";

julia> typeof(re)
Regex
\end{verbatim}

To check if a regex matches a string one can use the function
\texttt{occursin}. But \texttt{occursin} simply returns \texttt{true} or
\texttt{false} indicating whether a match for the given regex occurs in
the string. Commonly, however, one wants to know not just whether a
string matched, but also how it matched. To capture this information
about a match, use the \texttt{match} function instead.

\begin{verbatim}
julia> match(r"^\s*(?:#|$)", "# a comment")
RegexMatch("#")
\end{verbatim}

If the regular expression does not match the given string,
\texttt{match} returns \texttt{nothing}, a special value that does not
print anything at the interactive prompt. Other than not printing, it is
a completely normal value and one can test for it programmatically. If a
regular expression does match, the value returned by match is a
\texttt{RegexMatch} object.

\newpage

\hypertarget{functions}{%
\section{Functions}\label{functions}}

In \textbf{Julia}, a function is an object that maps a tuple of argument
values to a return value. \textbf{Julia} functions are not pure
mathematical functions, because they can alter and be affected by the
global state of the program. The basic syntex for defining functions in
\textbf{Julia} is:

\begin{verbatim}
function f(x, y))
  x - y
end
\end{verbatim}

This function accepts two arguments \texttt{x} and \texttt{y} and
returns the value of the last expression evaluated, which is
\texttt{x-y}.

In \textbf{Julia}, we can also define inline as
\texttt{f(x,y)\ =\ x\ -\ y}, called the ``assignment form''. In the
``assignment form'', the body of the function must be a single
expression, although it can be a compound expression.\\
In \textbf{Julia}, functions are objects in their own; they can be
assigned or spitted out as a return value of a function.

\hypertarget{argument-passing-behavior}{%
\subsection{Argument Passing Behavior}\label{argument-passing-behavior}}

\textbf{Julia} function arguments follow a convention sometimes called
``pass-by-sharing'', which means that values are not copied when they
are passed to functions. Function arguments themselves act as new
variable bindings so that objects they refer to are identical to the
passed values. Modifications to mutable values (such as \emph{Arrays})
made within a function will be visible to the caller. For eg.:

\begin{verbatim}
function f(x,y)
  x[1] = 3.9
  y = 7 + y
  return y
end
\end{verbatim}

In this code, the statement \texttt{x{[}1{]}\ =\ 3.9} mutates the object
\texttt{x,}, and this change will be visible in the array passed by the
caller for this argument. On the other hand, the assignment
\texttt{y\ =\ 7\ +\ y} changes the binding (``name'') \texttt{y} to
refer to a new value rather than mutatingj the original object; and
hence does not change the corresponding argument passed by the caller.

As a \emph{common convention} in \textbf{Julia} (not syntactically
required), such a function would typically be named \texttt{f!(x,y)}
rather than \texttt{f(x,y)}, as a visual reminder at the call site that
at least one of the arguments (often the first one) is being mutated.

\hypertarget{argument-type-declarations}{%
\subsection{Argument-type
declarations}\label{argument-type-declarations}}

You can declare the types of function arguements by appending
\texttt{::TypeName} to the arguemnt name, as usual for \emph{Type
Declarations} in \textbf{Julia}. Eg.:

\begin{verbatim}
fib(n::Integer) = n <= 2 ? one(n) :  fib(n-1) + fib(n-2)
\end{verbatim}

and the \texttt{::Integer} specification means that it will only be
callable when \texttt{n} is a subtype of the abstract \texttt{Integer}
type.

\hypertarget{the-return-keyword}{%
\subsection{The `return' keyword}\label{the-return-keyword}}

In \textbf{Julia}, the value returned by a function is the value of the
last expression evaluated, which, by default, is the last expression in
the body of the function definition. As an alternative, the
\texttt{return} keyword causes a function to return immediately,
providing an expression whose value is returned.

A return type can be specified in the function declaration using the
\texttt{::} oeprator. This converts the return value to the specified
type.

\begin{verbatim}
function g(x,y)::Int8
  return x*y
end
\end{verbatim}

This function will always return an \texttt{Int8} regardless of the
types of \texttt{x} and \texttt{y}.

For functions that do not need to return a value (functions used only
for some side effects), the \textbf{Julia} convention is to return the
value \texttt{nothing}.

\begin{verbatim}
function printx(x)
  println("x = $x")
  return nothing
end
\end{verbatim}

\hypertarget{operators-are-functions}{%
\subsection{Operators are Functions}\label{operators-are-functions}}

In \textbf{Julia}, most operators are just functions with support for
special syntax. The exceptions are operators with special evaluation
semantics like \texttt{\&\&} and \texttt{\textbar{}\textbar{}}. These
operators cannot be functions since \emph{Short-Circuit Evaluation}
requires that their operands are not evaluated before evaluation of the
operator. Hence codes such as \texttt{+(1,7,8)} or \texttt{*(9,7,5)} are
perfectly valid.

\hypertarget{operators-with-special-names}{%
\subsection{Operators with Special
Names}\label{operators-with-special-names}}

A few special expressions correspond to calls to functions with
non-obvious names.

\begin{longtable}[]{@{}lc@{}}
\toprule\noalign{}
Expression & Calls \\
\midrule\noalign{}
\endhead
\bottomrule\noalign{}
\endlastfoot
\texttt{{[}A\ B\ C\ ...\ {]}} & \texttt{hcat} \\
\texttt{{[}A;\ B;\ C;\ ...{]}} & \texttt{vcat} \\
\texttt{{[}A\ B;\ C\ D;\ ...\ {]}} & \texttt{hvcat} \\
\texttt{A\textquotesingle{}} & \texttt{adjoint} \\
\texttt{A{[}i{]}} & \texttt{getindex} \\
\texttt{A{[}i{]}\ =\ x} & \texttt{setindex!} \\
\texttt{A.n} & \texttt{getproperty} \\
\texttt{A.n\ =\ x} & \texttt{setproperty!} \\
\end{longtable}

\hypertarget{anonymous-functions}{%
\subsection{Anonymous Functions}\label{anonymous-functions}}

Functions in \textbf{Julia} are \emph{first-class objects}: they can be
assigned to variables, and called using the standard function call
syntax from the variable they have been assigned to. They can be used as
arguments, and they can be returned as values. They can also be created
anonymously, without being given a name, using either of these syntaxes:

\begin{verbatim}
julia> x -> x^2 + 2x - 1
#1 (generic function with 1 method)
julia> function (x)
         x^2 + 2x - 1
       end
#3 (generic function with 1 method)
\end{verbatim}

The primary use for anonymous functions is passing them to functions
which take other functions as arguments. A classic example of such a
function is \texttt{map} which applies a function to each value of an
array and returns a new array containing the resulting values.

\begin{verbatim}
julia> map(round, [1.3, 0.25, 1.74, 7.9])
4-element Vector{Float64}:
 1.0
 0.0
 2.0
 8.0
\end{verbatim}

\hypertarget{tuples}{%
\subsection{Tuples}\label{tuples}}

\textbf{Julia} has a built-in data structure called a tuple that is
closely related to function arguments and return values. A tuple is a
fixed-length container that can hold any values, but cannot be modified
(it is immutable). Tuples are constructed with commas and parentheses,
and can be accessed via indexing.

The components of tuples can optionally be named, in which case a named
tuple is constructed.

\hypertarget{destructuring-assignment-and-multiple-return-values}{%
\subsection{Destructuring Assignment and Multiple Return
Values}\label{destructuring-assignment-and-multiple-return-values}}

A comma-separated list of variables (optionally wrapped in parentheses)
can appear on the left side of an assignment: the value on the right
side is destructured by iterating over and assigning to each variable in
turn.

If only a subset of the elements of the iterator are required, a common
convention is to assign ignored elements to a variable consisting of
only underscores.

If the last symbol in the assignment list is suffixed by \texttt{...}
(known as slurping), then it will be assigned a collection or lazy
iterator of the remaining elements of the right-hand side iterator.

Slurping in assignments can also occur in any other position (requires
\textbf{Julia 1.9}). As opposed to slurping the end of a collection
however, this will always be eager.

\hypertarget{varargs-functions}{%
\subsection{Varargs Functions}\label{varargs-functions}}

It is often convenient to be able to write functions taking an arbitrary
number of arguments. Such functions are traditionally known as
``varargs'' functions, which is short for ``variable number of
arguments''. You can define a varargs function by following the last
positional argument with an ellipsis:

\begin{verbatim}
julia> bar(a,b,x...) = (a,b,x)
bar (generic function with 1 method)
\end{verbatim}

The variables \texttt{a} and \texttt{b} are bound to the first two
argument values as usual, and the variable \texttt{x} is bound to an
iterable collection of the zero or more values passed to bar after its
first two arguments.

\begin{verbatim}
julia> bar(1,2)
(1, 2, ())

julia> bar(1,2,3)
(1, 2, (3,))

julia> bar(1, 2, 3, 4)
(1, 2, (3, 4))

julia> bar(1,2,3,4,5,6)
(1, 2, (3, 4, 5, 6))
\end{verbatim}

\hypertarget{optional-arguments}{%
\subsection{Optional Arguments}\label{optional-arguments}}

It is often possible to provide sensible default values for function
arguments. This can save users from having to pass every argument on
every call. For example, the function \texttt{Date(y,\ {[}m,\ d{]})}
from \texttt{Dates} module constructs a \texttt{Date} type for a given
year \texttt{y}, month \texttt{m} and day \texttt{d}. However,
\texttt{m} and \texttt{d} arguments are optional and their default value
is 1. This behavior can be expressed concisely as:

\begin{verbatim}
julia> using Dates

julia> function date(y::Int64, m::Int64=1, d::Int64=1)
         err = Dates.validargs(Date, y, m, d)
         err === nothing || throw(err)
         return Date(Dates.UTD(Dates.totaldays(y, m, d)))
       end
date (generic function with 3 methods)
\end{verbatim}

Observe, that this definition calls another method of the Date function
that takes one argument of type \texttt{UTInstant\{Day\}}. With this
definition, the function can be called with either one, two or three
arguments, and 1 is automatically passed when only one or two of the
arguments are specified.

\hypertarget{keyword-arguments}{%
\subsection{Keyword Arguments}\label{keyword-arguments}}

Some functions need a large number of arguments, or have a large number
of behaviors. Remembering how to call such functions can be difficult.
Keyword arguments can make these complex interfaces easier to use and
extend by allowing arguments to be identified by name instead of only by
position.

For example, consider a function plot that plots a line. This function
might have many options, for controlling line style, width, color, and
so on. If it accepts keyword arguments, a possible call might look like
\texttt{plot(x,\ y,\ width=2)}, where we have chosen to specify only
line width. Notice that this serves two purposes. The call is easier to
read, since we can label an argument with its meaning. It also becomes
possible to pass any subset of a large number of arguments, in any
order.

\textbf{Functions with keyword arguments are defined using a semicolon
in the signature}:

\begin{verbatim}
function plotMe(x, y; style="solid", width=1, color="black")
  ###
end
\end{verbatim}

When the function is called, the semicolon is optional; one can either
call \texttt{plotMe(x,\ y,\ width=2)} or \texttt{plot(x,\ y;\ width=2)},
but the former style is more common. An explicit semicolon is required
only for passing varargs or computed keywords as described below.\\
Keyword argument default values are evaluated only when necessary (when
a corresponding keyword argument is not passed), and in left-to-right
order. Therefore default expressions may refer to prior keyword
arguments.

Keyword arguments can also be used in varargs functions.\\
Extra keyword arguments can be collected using \texttt{...}, as in
varargs functions.

\hypertarget{do-block-syntax-for-function-arguments}{%
\subsection{Do-Block Syntax For Function
Arguments}\label{do-block-syntax-for-function-arguments}}

Passing functions as arguments to other functions is a powerful
technique, but the syntax for it is not always convenient. Such calls
are especially awkward to write when the function argument requires
multiple lines. As an example, consider calling \texttt{map} on a
function with several cases:

\begin{verbatim}
map(x->begin
           if x < 0 && iseven(x)
             return 0
           elseif x == 0
             return 1
           else
             return x
           end
       end,
     [A, B, C])
\end{verbatim}

\textbf{Julia} provides a reserved word \texttt{do} for rewriting this
code more clearly:

\begin{verbatim}
map([A, B, C]) do x
  if x < 0 && iseven(x)
    return 0
  elseif x == 0
    return 1
  else
    return x
  end
end
\end{verbatim}

The \texttt{do\ x} syntax creates an anonymous function with argument
\texttt{x} and passes it as the first argument to \texttt{map}.
Similarly, \texttt{do\ a,b} would create a two-argument anonymous
function. Note that \texttt{do\ (a,b)} would create a one-argument
anonymous function, whose argument is a tuple to be deconstructed. A
plain \texttt{do} would declare that what follows is an anonymous
function of the form \texttt{()\ -\textgreater{}\ ...}. How these
arguments are initialized depends on the ``outer'' function; here,
\texttt{map} will sequentially set \texttt{x} to \texttt{A}, \texttt{B},
\texttt{C}, calling the anonymous function on each, just as would happen
in the syntax \texttt{map(func,\ {[}A,\ B,\ C{]})}. This syntax makes it
easier to use functions to effectively extend the language, since calls
look like normal code blocks. There are many possible uses quite
different from map, such as managing system state. For example, there is
a version of open that runs code ensuring that the opened file is
eventually closed:

\begin{verbatim}
open("outfile", "w") do io
  write(io, data)
end
\end{verbatim}

This is accomplished by the following definition:

\begin{verbatim}
function open(f::Function, args...)
  io = open(args...)
  try
    f(io)
  finally
    close(io)
  end
end
\end{verbatim}

Here, \texttt{open} first opens the file for writing and then passes the
resulting output stream to the anonymous function you defined in the
\texttt{do\ ...\ end} block. After your function exits, \texttt{open}
will make sure that the stream is properly closed, regardless of whether
your function exited normally or threw an exception.

\hypertarget{function-composition-and-piping}{%
\subsection{Function Composition and
Piping}\label{function-composition-and-piping}}

Functions in \textbf{Julia} can be combined by composing or piping
(chaining) them together. Function composition is when you combine
functions together and apply the resulting composition to arguments. You
use the function composition operator (\(\circ\)) to compose the
functions, so \((f\circ g)(args...)\) is the same as \(f(g(args...))\).

You can type the composition operator at the \texttt{REPL} and
suitably-configured editors using
\texttt{\textbackslash{}circ\textless{}tab\textgreater{}}.

Function chaining (sometimes called ``piping'' or ``using a pipe'' to
send data to a subsequent function) is when you apply a function to the
previous function's output. The pipe operator can also be used with
broadcasting, as \texttt{.\textbar{}\textgreater{}}, to provide a useful
combination of the chaining/piping and dot vectorization syntax. When
combining pipes with anonymous functions, parentheses must be used if
subsequent pipes are not to parsed as part of the anonymous function's
body.

\hypertarget{dot-syntax-for-vectorizing-functions}{%
\subsection{Dot Syntax for Vectorizing
Functions}\label{dot-syntax-for-vectorizing-functions}}

In technical-computing languages, it is common to have ``vectorized''
versions of functions, which simply apply a given function \texttt{f} to
each element of an array \texttt{A} to yield a new array via
\texttt{f(A)}. This kind of syntax is convenient for data processing,
but in other languages vectorization is also often required for
performance: \texttt{if} loops are slow, the ``vectorized'' version of a
function can call fast library code written in a low-level language. In
\textbf{Julia}, vectorized functions are not required for performance,
and indeed it is often beneficial to write your own loops, but they can
still be convenient. Therefore, any \textbf{Julia} function \texttt{f}
can be applied elementwise to any array (or other collection) with the
syntax \texttt{f.(A)}. For example, \texttt{sin} can be applied to all
elements in the vector \texttt{A} like so:

\begin{verbatim}
julia> A = [1.0, 2.0, 3.0]
3-element Vector{Float64}:
 1.0
 2.0
 3.0

julia> sin.(A)
3-element Vector{Float64}:
 0.8414709848078965
 0.9092974268256817
 0.1411200080598672
\end{verbatim}

Of course, you can omit the dot if you write a specialized ``vector''
method of \texttt{f} e.g.~via
\texttt{f(A::AbstractArray)\ =\ map(f,\ A)} and this is just as
efficient as \texttt{f.(A)}. The advantage of the \texttt{f.(A)} syntax
is that which functions are vectorizable need not be decided upon in
advance by the library writer. More generally, \texttt{f.(args...)} is
actually equivalent to \texttt{broadcast(f,\ args...)}, which allows you
to operate on multiple arrays, or a mix of arrays and scalars.\\
Keyword arguments are not broadcasted over, but are simply passed
through to each call of the function. For example,
\texttt{round.(x,\ digits=3)} is equivalent to
\texttt{broadcast(x\ -\textgreater{}\ round(x,\ digits=3),\ x)}.

Moreover, nested \texttt{f.(args...)} calls are fused into a single
broadcast loop. For example, \texttt{sin.(cos.(X))} is equivalent to
\texttt{broadcast(x\ -\textgreater{}\ sin(cos(x)),\ X)}, similar to
\texttt{{[}sin(cos(x))\ for\ x\ in\ X{]}} there is only a single loop
over \texttt{X}, and a single array is allocated for the result (In
contrast, \texttt{sin(cos(X))} in a typical ``vectorized'' language
would first allocate one temporary array for \texttt{tmp=cos(X)}, and
then compute \texttt{sin(tmp)} in a separate loop, allocating a second
array). This loop fusion is not a compiler optimization that may or may
not occur, it is a syntactic guarantee whenever nested
\texttt{f.(args...)} calls are encountered. Technically, the fusion
stops as soon as a ``non-dot'' function call is encountered; for
example, in \texttt{sin.(sort(cos.(X)))} the \texttt{sin} and
\texttt{cos} loops cannot be merged because of the intervening
\texttt{sort} function.

Finally, the maximum efficiency is typically achieved when the output
array of a vectorized operation is pre-allocated, so that repeated calls
do not allocate new arrays over and over again for the results. A
convenient syntax for this is \texttt{X\ .=\ ...}, which is equivalent
to \texttt{broadcast!(identity,\ X,\ ...)} except that, as above, the
\texttt{broadcast!} loop is fused with any nested ``dot'' calls. For
example, \texttt{X.=\ sin.(Y)} is equivalent to
\texttt{broadcast!(sin,\ X,\ Y)}, overwriting \texttt{X} with
\texttt{sin.(Y)} in-place. If the left-hand side is an array-indexing
expression, e.g.~\texttt{X{[}begin+1:end{]}\ .=\ sin.(Y)}, then it
translates to \texttt{broadcast!} on a view,
e.g.~\texttt{broadcast!(sin,\ view(X,\ firstindex(X)+1:lastindex(X)),\ Y)},
so that the left-hand side is updated in-place.

\newpage

\hypertarget{control-flow}{%
\section{Control Flow}\label{control-flow}}

\textbf{Julia} provides a variety of control flow constructs:

\begin{itemize}
\tightlist
\item
  \textbf{Compound Expressions}: \texttt{begin} and \texttt{;}
\item
  \textbf{Conditional Evaluation}: \texttt{if} \ldots{} \texttt{elseif}
  \ldots{} \texttt{else} and \texttt{?} (the ternary operator)
\item
  \textbf{Short-Circuit Evaluation}: Logical operators \texttt{\&\&} and
  \texttt{\textbar{}\textbar{}} and also chained comparisons.
\item
  \textbf{Repeated Evaluation}: loops like \texttt{while} or
  \texttt{for}.
\item
  \textbf{Exception Handling} like \texttt{try} \ldots{} \texttt{catch},
  \texttt{error} and \texttt{throw}.
\item
  \textbf{Tasks (Corountines)}: \texttt{yieldto}.
\end{itemize}

\textbf{Tasks} provide non local control flow, making it possible to
switch between temporarily-suspended computations. This is a powerful
construct: both exception handling and cooperative multitasking are
implemented in \textbf{Julia} using \textbf{Tasks}.

\hypertarget{compound-expressions}{%
\subsection{\texorpdfstring{\textbf{Compound
Expressions}}{Compound Expressions}}\label{compound-expressions}}

Compound expressions are basically a whole expresison composed of
multiple subexpressions in order returning the value of the last
subexpression as its value. There are two \textbf{Julia} constructs that
accomplish that: \texttt{begin} blocks and \texttt{;} chains. The value
of both compound expression constructs is that of the last
subexpression. Here's an example of a \texttt{begin} block:

\begin{verbatim}
julia> z = begin
         x = 1
         y = 7x^2 + 63
         sqrt(3x) + y
       end

71.73205080756888
\end{verbatim}

Since there are fairly small simple expresisons, they could've easily
been placed onto a single line, which is where the \texttt{;} chain
syntax comes in handy.

\begin{verbatim}
julia> z = (x=1; y = 7x^2 + 63; sqrt(3x) + y);
\end{verbatim}

Although it is typical, there is no requirement for the \texttt{begin}
blocks to be multiline or the \texttt{;} chains to be single lined.

\hypertarget{conditinal-evaluation.}{%
\subsection{\texorpdfstring{\textbf{Conditinal
Evaluation}.}{Conditinal Evaluation.}}\label{conditinal-evaluation.}}

Conditional evaluation allows portions of code to be evaluated or not
depending on the value of a boolean expression. The anatomy is the
following:

\begin{verbatim}
if contidion1
  action1
elseif condition2
  action2
.
.
.
elseif contitionk
  actionk
else
  alternative_action
end
\end{verbatim}

The \texttt{elseif} and \texttt{else} blocks are optional.\\
\texttt{if} blocks do not introduce local scopre, they're then
\emph{leaky}: new variables defined inside the \texttt{if} clauses can
be used after the \texttt{if} block even if they weren't defined before.

\textbf{WARNING}: Unlike \texttt{C}, \texttt{MATLAB}, \texttt{Perl},
\texttt{Python} and \texttt{Ruby}, but like \texttt{Java} and few other
stricter typed languages, it is an error if the value of a conditional
expression is anything but \texttt{true} or \texttt{false}.

The so-callled \emph{ternary operator} \texttt{?\ :} is closely related
to the \texttt{if-elseif-else} syntax but is used where a conditional
choice between single expression values is required, as opposed to
conditional execution of longer blocks of code. It gets its name from
being the only operator in most languages taking three operands. It's
syntax is as follows:

\begin{verbatim}
a ? b : c
\end{verbatim}

Where \texttt{a} is the condition, \texttt{b} the expression that's
evaluated if \texttt{a} evaluated to \texttt{true} and \texttt{c} the
one that's evaluatend otherwise.

\hypertarget{short-circuit-evaluation.}{%
\subsection{\texorpdfstring{\textbf{Short-Circuit
Evaluation}.}{Short-Circuit Evaluation.}}\label{short-circuit-evaluation.}}

The \texttt{\&\&} and \texttt{\textbar{}\textbar{}} operators in
\textbf{Julia} correspond to logicall \emph{and} and \emph{or}
operations. And as in some languages they're \emph{short-circuited}
mieaning that they do not necessarily evaluate their remaining arguments
once the truth value of the whole expression if fully determined.

\hypertarget{repeated-evaluation-loops}{%
\subsection{\texorpdfstring{\textbf{Repeated Evaluation:
loops}}{Repeated Evaluation: loops}}\label{repeated-evaluation-loops}}

There are two constructs for repeated evaluation of expressions: the
\texttt{while} loop and the \texttt{for} loop. Their syntaxes are as
follows (resp.):

\begin{verbatim}
while condition
  action
end
\end{verbatim}

\begin{verbatim}
for conditioned_expression
  action
end
\end{verbatim}

\textbf{WARNING}: The \texttt{for} loop instroduces a new scope.

\texttt{for} loops can be exited early by using \texttt{break}, or a
step can be skipped using \texttt{continue}. Note that multiple
containers can be iterated over at the same time in a single
\texttt{for} loop using the \texttt{zip} function.

\hypertarget{exception-handling.}{%
\subsection{\texorpdfstring{\textbf{Exception
Handling}.}{Exception Handling.}}\label{exception-handling.}}

When an unexpected condition occurs, a function may be unable to return
a reasonable value to its caller. In such cases, it may be best for the
exceptional condition to either terminate the program while printing a
diagnostic error message, or if the programmer has provided code to
handle such exceptional circumstances then allow that code to take the
appropriate action.

Thre are built-in exceptions in \textbf{Julia}, they're explicitly
thrown using the function \texttt{throw} and interrupt the normal flow
of control.

\begin{verbatim}
f(x) =  x > 0 ? 1 : throw(DomainError(x, "Damn! Fuck it man, x must be positive"))
\end{verbatim}

\begin{longtable}[]{@{}c@{}}
\toprule\noalign{}
Exception \\
\midrule\noalign{}
\endhead
\bottomrule\noalign{}
\endlastfoot
ArgumentError \\
BoundsError \\
CompositeException \\
DimensionMismatch \\
DivideError \\
DomainError \\
EOFError \\
ErrorException \\
InexactError \\
InitError \\
InterruptException \\
InvalidStateException \\
KeyError \\
LoadError \\
OutOfMemoryError \\
ReadOnlyMemoryError \\
RemoteException \\
MethodError \\
OverflowError \\
Meta.ParseError \\
SystemError \\
TypeError \\
UndefRefError \\
UndefVarError \\
StringIndexError \\
\end{longtable}

Note that the \texttt{error} function is used to produce an
\texttt{ErrorException} that interrupts the normal flow of control.

The \texttt{try/catch} statement allows for \texttt{Exceptions} to be
tested for, and for the graceful handling of things that may ordinarily
break the application.

\begin{verbatim}
try
  primary_action
catch an_error_variable_name
  actions_to_handle_the_fucking_bugs
else
  do_something_default
finally
  always_do_this_shit_at_the_end
end
\end{verbatim}

The \emph{finally} clause is usefult in codes which perform state
changes or use resources like files, there is typically clean-up work
(closing files \ldots) that need to be done when the code is finished.

The \texttt{finally} keyword provides a way to run some code when a
given block of code exits, regardless of how it exits.

\newpage

\hypertarget{scope-of-variables}{%
\section{Scope of Variables}\label{scope-of-variables}}

The \emph{scope of a variable} is the region of code within which a
variable is accessible. Variable scoping helps avoid variable naming
conflicts. The concept is intuitive: two functions can both have
arguments called \texttt{x} without the two \texttt{x} referring to the
same thing. Similarly, there are many other cases where different blocks
of code can use the same name without referring to the same thing. The
rules for when the same variable name does or doesn't refer to the same
thing are called scope rules.

The scope of a variable cannot be an arbitrary set of source lines,
instead, it will always line up with one of these blocks. There are two
main types of scopes in \textbf{Julia}, \emph{global} and \emph{local}
scope. In \textbf{julia} there's also a distinction between constructs
which introduce a \emph{hard scope} and those which only introduce a
\emph{soft scope} which affect wether shadowing a global variable by the
name is allowed or not.

\hypertarget{scope-constructs}{%
\subsection{Scope Constructs}\label{scope-constructs}}

The constructs introducing scope blocks are:

\begin{longtable}[]{@{}
  >{\centering\arraybackslash}p{(\columnwidth - 4\tabcolsep) * \real{0.3333}}
  >{\centering\arraybackslash}p{(\columnwidth - 4\tabcolsep) * \real{0.3333}}
  >{\centering\arraybackslash}p{(\columnwidth - 4\tabcolsep) * \real{0.3333}}@{}}
\toprule\noalign{}
\begin{minipage}[b]{\linewidth}\centering
\textbf{Construct}
\end{minipage} & \begin{minipage}[b]{\linewidth}\centering
\textbf{Scope Type}
\end{minipage} & \begin{minipage}[b]{\linewidth}\centering
\textbf{Allowed Within}
\end{minipage} \\
\midrule\noalign{}
\endhead
\bottomrule\noalign{}
\endlastfoot
module, baremodule & global & global \\
struct & local (soft) & global \\
\texttt{for}, \texttt{while}, \texttt{try} & local (soft) & global,
local \\
macro & local (hard) & global \\
functions, \texttt{do} blocks, \texttt{let} blocks, comprehensions,
generators & local (hard) & global, local \\
\end{longtable}

Notably missing from this table are \texttt{begin} blocks and
\texttt{if} blocks, which do not introduce new scopes. The three types
of scopes (local(soft), local(hard) and global) follow somewhat
different rules.

\textbf{Julia} uses lexical scoping meaning that a function's scope does
not inherit from its caller scope but from the scope in which the
function was defined. For eg. in this code

\begin{verbatim}
julia> module Bar
         x = 1
         foo() = x
       end;
\end{verbatim}

the \texttt{x} inside \texttt{foo} refers to the \texttt{x} in the
global scope of its module \texttt{Bar} and not an eventual \texttt{x}
where the function may be used.

\hypertarget{global-scope}{%
\subsection{Global Scope}\label{global-scope}}

Each module introduces a new global scope, separated from the global
scope of all other modules (there is no all-encompassing global scope).
Modules can introduce variables of other modules into their scope
through the \texttt{using} or \texttt{import} statements or through
qualified access using the dot-notation, i.e. each module is a so-called
namespace as well as a first-class data structure associating names with
values. Note that while variable bindings can be read externally, they
can only be changed within the module to which they belong. As an escape
hatch, you can always evaluate code inside that module to modify a
variable; this guarantees, in particular, that module bindings cannot be
modified externally by code that never calls eval.

The syntax to make a module (say \texttt{A}) available is
\texttt{import\ ..A}.

If a top-level expression contains a variable declaration with keyword
\texttt{local}, then that variable is not accessible outside that
expression. The variable inside the expression does not affect global
variables of the same name.

\hypertarget{local-scope}{%
\subsection{Local Scope}\label{local-scope}}

A new local scope is introduced by most code blocks. If such a block is
syntactically nested inside of another local scope, the scope it creates
is nested inside of all local scopes that it appears within, which are
all ultimately nested inside of the global scope of the module in which
the code is evaluated.

Variables in outer scopes are visible from any scope they contain,
meaning that they can be read and written in inner scopes, unless there
is a local variable with the same name that \underline{shadows} the
outer variable of the same name. This is true even if the outer local is
declared after (in the sense of textually below) an inner block. When we
say that a variable exists in a given scope, this means that a variable
by that name exists in any of the scopes that the current scope is
nested inside of, including the current one.

Some programming languages require explicitly declaring new variables
before using them explicit declaration works in \textbf{Julia} too: in
any local scope, writing \texttt{local\ x} declares a new local variable
in that scope, regardless of whether there is already a variable named
\texttt{x} in an outer scope or not. Declaring each new variable like
this is somewhat verbose and tedious, however, so \textbf{Julia}, like
many other languages, considers assignment to a variable name that
doesn't already exist to implicitly declare that variable. If the
current scope is global, the new variable is global; if the current
scope is local, the new variable is local to the innermost local scope
and will be visible inside of that scope but not outside of it. If you
assign to an existing local it always updates that existing local: you
can only shadow a local by explicitly declaring a new local in a nested
scope with the \texttt{local} keyword. In particular, this applies to
variables assigned in inner functions, which may surprise users coming
from Python where assignment in an inner functin creates a new local
unless the variable is explicitly declared to be non-local.

When \texttt{x\ =\ \textless{}value\textgreater{}} occurs in a local
scope, \textbf{Julia} applies the following rules to decide what the
expression means based on where the assignment expression occurs and
what \texttt{x} already refers to at that location:

\begin{enumerate}
\def\labelenumi{\arabic{enumi}.}
\tightlist
\item
  \textbf{Existing local}: If \texttt{x} is already a local variable,
  then the existing local \texttt{x} is assigned.
\item
  \textbf{Hard scope}: If \texttt{x} is not already a local variable and
  assignment occurs inside of any hard scope construct (i.e.~within a
  \texttt{let} block, function or macro body, comprehension, or
  generator), a new local named \texttt{x} is created in the scope of
  the assignment.
\item
  \textbf{Soft Scope}: If \texttt{x} is not already a local variable and
  all of the scope constructs containing the assignment are soft scopes
  (loops, \texttt{try}/\texttt{catch} blocks, or \texttt{struct}
  blocks), the behavior depends on whether the global variable
  \texttt{x} is defined:

  \begin{itemize}
  \tightlist
  \item
    If global \texttt{x} is undefined, a new local named \texttt{x} is
    created in the scope of the assignment.
  \item
    If global \texttt{x} is defined, the assignment is considered
    ambiguous: *in non-interactive contexts (files, eval), an ambiguity
    warning is printed and a new local is created; in interactive
    contexts (REPL, notebooks), the global variable \texttt{x} is
    assigned.
  \end{itemize}
\end{enumerate}

\hypertarget{let-blocks}{%
\subsubsection{\texorpdfstring{\texttt{let}
blocks}{let blocks}}\label{let-blocks}}

\texttt{let} statements create a new hard scope block and introduce new
variable bindings each time they run. The variable need not be
immediately assigned:

\begin{verbatim}
julia> var1 = let x
         for i in 1:7
           (i == 4) && (x = i; break)
         end
         x
       end

4
\end{verbatim}

Whereas assignments might reassign a new value to an existing value
location, \texttt{ket} always creates a new location. This difference is
usually not important, and is only detectable in the case of variables
that outlive their scope via closures. The \texttt{let} syntax accepts a
comma-separated series of assignments and variable names.

The assignments are evaluated in order, with each right-hand side
evaluated in the scope before the new variable on the left-hand side has
been introduced. Therefore it makes sense to write something like
\texttt{let\ x\ =\ x} since the two \texttt{x} variables are distinct
and have separate storage. Here is an example where the behavior of
\texttt{let} is needed:

\begin{verbatim}
julia> Fs = Vector{Any}(undef, 2); i = 1;

julia> while i <= 2
         Fs[i] = () -> i
         global i += i
       end

julia> Fs[1]()
3

julia> Fs[2]()
3
\end{verbatim}

Here we create and sotre two closures that return variable \texttt{i}.
However, it is always the same variable \texttt{i}, so the two closures
behave identically. We can use \texttt{let} to create a new binding for
\texttt{i}:

\begin{verbatim}
julia> Fs = Vector{Any}(undef, 2); i = 1;


julia> while i <= 2
         let i = i
           Fs[i] = ()->i
         end
         global i += i
       end

julia> Fs[1]()
1

julia> Fs[2]()
2
\end{verbatim}

Since the \texttt{begin} construct does not introduce a new scope, it
can be useful to use a zero-argument \texttt{let} to juste introduce a
new scope block without any new bindings immediately.

\begin{verbatim}
julia> let
          local x = 1
          let
            local x = 2
          end
          x
       end
1
\end{verbatim}

Since \texttt{let} introduces a new scope block, the inner local
\texttt{x} is a different variable than the outer local \texttt{x}. This
particular example is equivalent to:

\begin{verbatim}
julia> let x = 1
         let x = 2
         end
         x
       end
1
\end{verbatim}

\hypertarget{loops-and-comprehensions}{%
\subsubsection{Loops and
Comprehensions}\label{loops-and-comprehensions}}

In loops and \emph{comprehensions}, new variables introduced in their
body scopes are freshly allocated for each loop iteration, as if the
loop body were surrounded by a \texttt{let} block, as demonstrated by
this example:

\begin{verbatim}
julia> Fs = Vector{Any}(undef, 2);

julia> for j = 1:2
         Fs[j] = ()->j
       end

julia> Fs[1]()
1

julia> Fs[2]()
2
\end{verbatim}

A \texttt{for} loop or comprehension iteration variable is always a new
variable.

\begin{verbatim}
julia> function f()
         i = 0
         for i = 1:3
           # empty
         end
         return i
       end;

julia> f()
0
\end{verbatim}

However, it is occasionally useful to reuse an existing local variable
as the iteration variable. This can be done conveniently by adding the
keyword \texttt{outer}:

\begin{verbatim}
julia> function f()
         i = 0
         for outer i = 1:3
           # empty
         end
         return i
       end;

julia> f()
3
\end{verbatim}

\hypertarget{constants}{%
\subsection{Constants}\label{constants}}

A common use of variables is giving names to specific, unchanging
values. Such variables are only assigned once. This intent can be
conveyed to the compiler using the \texttt{const} keyword.

\begin{verbatim}
julia> const e = 1.71828182845904523536;


julia> const pi = 3.14159265358979323846;
\end{verbatim}

Multiple variables can be declared in a single \texttt{const\ t}
statement:

\begin{verbatim}
julia> const a, b = 1, 2
(1, 2)
\end{verbatim}

The const declaration should only be used in global scope on globals. It
is difficult for the compiler to optimize code involving global
variables, since their values (or even their types) might change at
almost any time. If a global variable will not change, adding a
\texttt{const} declaration solves this performance problem.

Local constants are quite different. The compiler is able to determine
automatically when a local variable is constant, so local constant
declaratins are not necessary, and in fact are currently not supported.
Special top-level assignments, such as those performed by the
\texttt{function} and \texttt{struct} keywords, are constant by default.

Note that \texttt{const} only affects the variable binding; the variable
may be found to a mutable object (such as an array), and that object may
still be modified. Additionally when ones tries to assign a value to a
variable that is declared constant the following scenarios are possible.

\newpage

\hypertarget{types}{%
\section{Types}\label{types}}

Type systems have traditionally fallen into two quite different camps:
\textbf{static type} systems, where every program expression must have a
type computable before the execution of the program, and dynamic type
systems, where nothing is known about types until run time, when the
actual values manipulated by the program are available. Object
orientation allows some flexibility in statically typed languages by
letting code be written without the precise types of values being known
at compile time. The ability to write code that can operate on different
typejs is called polymorphism. All code in classic dynamically typed
languages is polymorphic: only by explicitly checking types, or when
objects fail to support operations at run-time, are the types of any
values ever restricted.

\textbf{Julia} type system is dynamic, but it's allowed to indicate
specific types. The default behavior in \textbf{Julia} when types are
omitted is to allow values to be of any type. Thus, one can write many
useful \textbf{Julia} functions without ever explicitly using types.
When additional expressiveness is needed however, it is easy to
gradually introduce explicit type annotations into previously
``untyped'' code. Adding annotations serves three primary purposes: to
take advantage of \textbf{Julia}'s powerful multiple-dispatch mechanism,
to improve human readability, and to catch programmer errors.

\textbf{Julia} is, regarding its type system, \emph{dynamic},
\emph{nominative} and \emph{parametric}.

Generic types can be parameterized, and the hierarchical relationships
between types are explicitly declared, rather than implied by compatible
structure. One particularly distinctive feature of \textbf{Julia}'s type
system is that concrete types may not subtype each other: all concrete
types are final and may only have abstract types as their supertypes.
While this might at first seem unduly restrictive, it has many
beneficial consequences with surprisingly few drawbacks. it turns out
that being able to inherit behavior is much more important than being
able to inherit structure, and inheriting both causes significant
difficulties in traditional objects-oriented languages. Other high-level
aspect of \textbf{Julia}'s type system that should be mentioned up front
are:

\begin{itemize}
\tightlist
\item
  There is no division between object and non-object values: all values
  in \textbf{Julia} are true objects having a type that belongs to a
  single, fully connected type graph, all nodes of which are equally
  first-class as types.
\item
  There is no meaningful concept os a ``compile-time type'': the only
  type a value has is its actual type when the program is running. This
  is called a ``run-time type'' in object-oriented languages where the
  combination of static compilation with polymorphism makes this
  distinction significant.
\item
  Only values, not variables, have types. Variables are simply names
  bound to values, although for simplicity we may ``type of a varibale''
  as shorthand for ``type of the value to which a variable refers''.
\item
  Both abstract and concrete types can be parameterized by other types.
  They can also be parameterized by symbols, by values of any type for
  which \texttt{isbits} returns true (essentially, things like numbers
  and bools that are stored like \textbf{C} types or \emph{structs} with
  no pointers to other objects), and also by tuples thereof. Type
  parameters may be omitted when they do not need to be referenced or
  restricted.
\end{itemize}

\hypertarget{type-declarations}{%
\subsection{Type Declarations}\label{type-declarations}}

The \texttt{::} operator can be used to attach type annotations to
expressions and variables in programs. When appended to an expression
computing a value, the \texttt{::} operator is read as ``is an instance
of''. It can be used anywhere to assert that the value of the expression
on the left is an instance of the type on the right. When appended to a
variable on the left-hand side of an assignment, or as part of a local
declaration, the \texttt{::} operator means something a bit different:
it declares the variable to always have the specified type, like a type
declaration in a statically-typed language such as \textbf{C}.

Declarations can also be attached to function definitions:

\begin{verbatim}
function sinc(x)::Float64
  if x == 0
    return 1
  end
  return sin(pi*x)/(pi*x)
end
\end{verbatim}

Returning from this function behaves just like an assignment to a
variable with a declared type: the value is always converted to
\texttt{Float64}.

\hypertarget{abstract-types}{%
\subsection{Abstract Types}\label{abstract-types}}

Abstract types cannot be instantiated, and serve only as nodes in the
type graph, thereby describing sets of related concrete types which are
their descendants. Abstract types are the backbone of the type system:
they form the conceptuall hierarchy which makes \textbf{Julia}'s type
system more than just a collection of object implementations. Abstract
types allow the construction of a hierarchy of types, providing a
context into which concrete types can fit. This allows you, for example,
to easily program to any type that is an integer, without restricting an
algorithm to a specific type of integer.

Abstract types are declared using the \texttt{abstract\ type} keyword.
The general syntaxes for declaring an abstract type are:

\begin{verbatim}
abstract type <<name>> end
abstract type <<name>> <: <<supertype>> end
\end{verbatim}

The \texttt{abstract\ type} keyword introduces a new abstract type whose
name is given by
\texttt{\textless{}\textless{}name\textgreater{}\textgreater{}}. This
name can be optionally followed by \texttt{\textless{}:} and an
already-existing type, indicating that the newly declared abstract type
is a subtype of this ``parent'' type. When no supertype is given, the
default supertype if \texttt{Any} (a predefined abstract type that all
objects are instances of and all types are subtypes of. In type theory,
\texttt{Any} is commonly called ``top'' because it is at the apex of the
type graph. \textbf{Julia} also has a predefined abstract ``bottom''
type, at the nadir of the type graph which is written as
\texttt{Union\{\}}. It is the exact opposite of \texttt{Any}: no object
is an instance of \texttt{Union\{\}} and all types are supertypes of
\texttt{Union\{\}}.

\begin{verbatim}
abstract type Number end
abstract type Real          <: Number end
asbtract type AbstractFloat <: Read end
abstract type Integer       <: Real end
abstract type Signed        <: Integer end
abstract type Unsigned      <: Integer end
\end{verbatim}

The \texttt{Number} type is a direct chid type of \texttt{Any}, and
\texttt{Real} is its child. In turn, \texttt{Real} has two children (it
has more, but only two are shown here): \texttt{Integer} and
\texttt{AbstractFloat}.

The \texttt{\textless{}:} operator in general means ``is a subtype of'',
and, used in declarations like those above, declares the right-hand type
to be an immediate supertype of the newly declared type. It can also be
used in expressions as subtype operator which returns \texttt{true} when
its left operand is a subtype of its right operand:

\begin{verbatim}
julia> Integer <: Number
true

julia> Integer <: AbstractFloat
false
\end{verbatim}

An important use of abstract types is to provide default impementations
for concrete types. To give a simple example, consider:

\begin{verbatim}
function myplus(x,y)
  x+y
end
\end{verbatim}

The first thing to note is that the above argument declarations are
equivalent to \texttt{x::Any} and \texttt{y::Any}. When this function is
invoked, say as \texttt{myplus(2,5)}, the dispathcer chooses the most
specific method named \texttt{myplus} that matches the given arguments.
Assuming no method more specific than the above is found, \textbf{Julia}
next internally defines and compiles a method called \texttt{myplus}
specifically for two \texttt{Int} arguments based on the generic
function given above, i.e., it implicitly defines and compiles:

\begin{verbatim}
function myplus(x::Int, y::Int)
  x+y
end
\end{verbatim}

\hypertarget{primitive-types}{%
\subsection{Primitive Types}\label{primitive-types}}

\begin{quote}
\begin{quote}
\textbf{WARNING} It is almost always preferable to wrap an existing
primitive type in a new composite type than to define your own primitive
type. This functionality exists to allow \textbf{Julia} to bootstrap the
standard primitive types that LLVM supports. Once they are defined,
there is very little reason to define more.
\end{quote}
\end{quote}

A primitive type is a concrete type whose data consists of plain old
bits. Classic examples of primitive types are integers and
floating-point values. Unlike most languages, \textbf{Julia} lets you
declare your own primitive types, rather than providing only a fixed set
of built-in ones. In fact, the standard primitive types are all defined
in the language itself:

\begin{verbatim}
primitive type Float16 <: AbstractFloat 16 end
primitive type Float32 <: AbstractFloat 32 end
primitive type Float64 <: AbstractFloat 64 end

primitive type Bool <: Integer 8 end
primitive type Char <: AbstractChar 32 end

primitive type Int8 <: Signed 8 end
primitive type UInt8 <: Unsigned 8 end
primitive type Int16 <: Signed 16 end
primitive type UInt16 <: Unsigned 16 end
primitive type Int32 <: Signed 32 end
primitive type UInt32 <: Unsigned 32 end
primitive type Int64 <: Signed 64 end
primitive type UInt64 <: Unsigned 64 end
primitive type Int128 <: Signed 128 end
primitive type UInt128 <: Unsigned 128 end
\end{verbatim}

The general syntaxes for declaring a primitive type are:

\begin{verbatim}
primitive type <<name>> <<bits>> end
primitive type <<name>> <: <<supertype>> <<bits>> end
\end{verbatim}

The number of bits indicates how much storage the type requires and the
name gives the new type a name. A primitive type can optionally be
declared to be a subtype of some supertype. If a supertype is omitted,
then the type defaults to having \texttt{Any} as its immediate
supertype. The declaration of \texttt{Bool} above therefore means that a
boolean value takes eight bits to store, and has \texttt{Integer} as its
immediate supertype. Currently, only sizes that are multiples of 8 bits
are supported and you are likely to experience LLVM bugs with sizes
other than those used above. Therefore, boolean values, although they
really need just a single bit, cannot be declared to be any smaller than
eight bits.

\hypertarget{composite-types}{%
\subsection{Composite Types}\label{composite-types}}

Composite types are called records, structs, or objects in various
languages. A composite type is a collection of named fields, an instance
of which can be treated as a single value. In many languages, composite
types are the only kind of user-definable type, and they are by far the
most commonly used user-defined type in \textbf{Julia} as well. In
mainstream object oriented languages, such as C++, Java, Python and
Ruby, composite types also have named functions associated with them,
and the combination is called an ``object''. In purer object-oriented
languages, such as Ruby or Smalltalk, all values are objects whether
they are composites or not. In less pure object oriented languages,
including C++ and Java, some values, such as integers and floating-point
values, are not objects, while instances of user-defined composite types
are true objects with associated methods. In Julia, all values are
objects, but functions are not bundled with the objects they operate on.

This is necessary since \textbf{Julia} chooses which method of a
function to use by multiple dispatch, meaning that the types of all of a
function's arguments are considered when selecting a method, rather than
just the first one (see \emph{Methods} for more information on methods
and dispatch). Thus, it would be inappropriate for functions to
``belong'' to only their first argument. Organizing methods into
function objects rather than havng named bags of methods ``inside'' each
object ends up being a hihly beneficial aspect of the language design.

Composite types are introduced with the \texttt{struct} keyword followed
by a block of field names, optionally annotated with types using the
\texttt{::} operator.

\begin{verbatim}
julia> struct Foo
         bar
         baz::Int
         qux::Float64
       end
\end{verbatim}

Fields with no type annotation default to \texttt{Any}, and can
accordingly hold any type of value. New objects of type \texttt{Foo} are
created by applying the \texttt{Foo} type object like a function to
values for its fields:

\begin{verbatim}
julia> foo = Foo("Hello, world.", 23, 1.5)
Foo("Hello, world.", 23, 1.5)

julia> typeof(foo)
Foo
\end{verbatim}

When a type is applied like a function it is called a constructor.

Two constructors are generated automatically (these are called default
constructors). One accepts any arguments and calls \texttt{convert} to
convert them to the types of the fields, and the other accepts arguments
that match the field types exactly. The reason both of these are
generated is that this makes it easier to add new definitions without
inadvertently replacing a default constructor.

One can find a list of field names using the \texttt{fieldnames}
function.

\begin{verbatim}
julia> fieldnames(Foo)
(:bar, :baz, :qux)
\end{verbatim}

One can access the field values of a composite object using the
traditional \texttt{foo.bar} notation.

Composite objects declared with \texttt{struct} are immutable, they
cannot be modified after construction.

An immutable object might contain mutable objects, such as arrays, as
fields. Those contained objects will remain mutable; only the fields of
the immutable object itself cannot be changed to point to different
objects. Where required, mutable composite objects can be declared with
the keyword \texttt{mutable\ struct}.

\hypertarget{mutable-composite-types}{%
\subsection{Mutable Composite Types}\label{mutable-composite-types}}

If a composite type is declared with \texttt{mutable\ struct} instead if
\texttt{struct}, then instances of it can be modified:

\begin{verbatim}
julia> mutable struct Bar
         baz
         qux::Float64
       end

julia> bar = Bar("Hello", 1.5);

julia> bar.qux = 2.0
2.0

julia> bar.baz = 1//2
1//2
\end{verbatim}

An extra interface between the fields and the user can be provided
through \emph{instance Properties}. This grants more control on what can
be accessed and modified using the \texttt{bar.baz} notation.

In order to support mutation, such objects are generally allocated on
the heap, and have stable memory addresses. A mutable object is like a
little container that might hold different values over time, and so can
only be reliably identified with its address. In constrast, an instance
of an immutable type is associated with specific field values (the field
values alone tell you everything about the object). In deciding whether
to make a type mutable, ask whether two instances with the same field
values would be considered identical, or ifthey might need to change
independently over time. If they would be considered identical, the type
should probably be immutable.

To recap, two essential properties define \emph{immutability} in
\textbf{Julia}:

\begin{itemize}
\tightlist
\item
  It is not permitted to modify the value of an immutable type.

  \begin{itemize}
  \tightlist
  \item
    For bits types this means that the bit pattern of a value once set
    will never cha,ge and that value is the identity of a bits type.
  \item
    For composite types, this means that identity of the values of its
    fields will never change. When the fields are bits types, that means
    their bits will never change, for fields whose values are mutable
    types like arrays, that means the fields will always refer to the
    same mutable value even though that mutable value's content may
    itself be modified.
  \end{itemize}
\item
  An objects with an immutable type may be copied freely by the compiler
  since its immutability makes it impossible to programmatically
  distinguish between the original objects and a copy.

  \begin{itemize}
  \tightlist
  \item
    In particular, this means that small enough immutable values like
    integers and floats are typically passed to functions in registers
    (or stack allocated).
  \item
    Mutable values, on the other hand are hea⁻allocated and passed to
    functions as pointers to heap-allocated values except in cases where
    the compiler is sure that there's no way to tell that this is not
    what is happening.
  \end{itemize}
\end{itemize}

Once declared mutable, some fields might be declared immutable using
\texttt{const}.

\hypertarget{declared-types}{%
\subsection{Declared Types}\label{declared-types}}

The three kinds of types (abstract, primitive, composite) discussed in
the previous sections are actually all closely related. They share the
same key properties:

\begin{itemize}
\tightlist
\item
  They are explicitly declared.
\item
  They have names.
\item
  They have explicitly declared supertypes.
\item
  They may have parameters.
\end{itemize}

Because of these shared properties, these types are internally
represented as instances of the same concept: \texttt{DataType}. Which
is the tupe of any of these types:

\begin{verbatim}
julia> typeof(Real)
DataType

julia> tyepof(Int)
DataType
\end{verbatim}

\hypertarget{type-unions}{%
\subsection{Type Unions}\label{type-unions}}

A type union is a special abstract type which includes as objects all
instances of any of its argument types, constructed using the special
\texttt{Union} keyword:

\begin{verbatim}
julia> IntOrString = Union{Int,AbstractString}
Union{Int64, AbstractString}

julia> 1 :: IntOrString
1

julia> "Hi" :: IntOrString
"Hi"

julia> 1.0 :: IntOrString
ERROR: TypeError: in typeassert, expected Union{Int64, AsbtractString}, got a value of type Float64
\end{verbatim}

A particularly useful case of a \texttt{Union} type is
\texttt{Union\{T,\ Nothing\}}, where \texttt{T} can be any type and
\texttt{Nothing} is the singleton type whose only instance is the object
\texttt{nothing}.

\hypertarget{parametric-types}{%
\subsection{Parametric types}\label{parametric-types}}

An important and powerful feature of \textbf{Julia}'s type system is
that it is parametric: types can take parameters, so that type
declarations actually introduce a whole family of new types (one for
each possible combination of parameter values).

We will note, however, that because \textbf{Julia} is a dynamically
typed language and doesn't need to make all type decisions at compile
time, many traditional difficulties encountered in static parametric
type systems can be relatively easily handled.

\hypertarget{parametric-composite-types}{%
\subsubsection{Parametric Composite
Types}\label{parametric-composite-types}}

Type parameters are introduced immediately after the type name,
surrounded by curly braces:

\begin{verbatim}
julia> struct Point{T}
         x::T
         y::Y
         z::T
       end
\end{verbatim}

This declaration defines a new paramaetric type, \texttt{Point\{T\}},
holding two ``coordinates'' of type \texttt{T}. What, one may ask, is
\texttt{T} ? Well, that's precisely the point of parametric types: it
can be any type at all (or a value of any bits type, actually, although
here it's clearly used as a type). \texttt{Point\{Float64\}} is a
concrete type equivalent to the type defined by replacing \texttt{T} in
the definition of \texttt{Point} with \texttt{Float64}. Thus, this
single declaration actually declares an unlimited number of
types\ldots{} each of these are then usable concrete types.

\textbf{WARNING}, even though \texttt{Float64\ \textless{}:\ Real}, we
do not have \texttt{Point\{Float64\}\ \textless{}:\ Point\{Real\}}. What
holds is that \texttt{Point\{Float64\}\ \textless{}:\ Point}.

\hypertarget{parametric-abstract-types}{%
\subsubsection{Parametric Abstract
Types}\label{parametric-abstract-types}}

Parametric abstract type declarations declare a collection of abstract
types, in much the same way:

\begin{verbatim}
julia> asbtract type Pointy{T} end
\end{verbatim}

With this declaration, \texttt{Pointy\{T\}} is a distinct abstract type
for each type or integer value of \texttt{T}. As with parametric
composite types, each such instance is a subtype of \texttt{Pointy}.

There are situations where it may not make sense for type parameters to
range freely over all possible types. In such situations, one can
constrain the range of \texttt{T}:

\begin{verbatim}
julia> abstract type Pointy{T<:Real} end
\end{verbatim}

\hypertarget{tuple-types}{%
\subsubsection{Tuple Types}\label{tuple-types}}

Tuples are an abstraction of the arguments of a function (without the
function itself). The salient aspects of a function's arguments are
their order and their types. Therefore a tuple type is similar to a
parameterized immutable type where each parameter is the type of one
field. For example, a 2-element tuple type resembles the following
immutable type:

\begin{verbatim}
struct Tuple2{A,B}
  a::A
  b::B
end
\end{verbatim}

However, there are three key differences:

\begin{itemize}
\tightlist
\item
  Tuple types may have any number of parameters.
\item
  Tuple types are covariant in their parameters: \texttt{Tuple\{Int\}}
  is a subtype of \texttt{Tuple\{any\}}. Therefor \texttt{Tuple\{Any\}}
  is considered an abstract type, and tuple are only concrete if their
  parameters are.
\item
  Tuples do not have field names; fields are only accessed by index.
\end{itemize}

Tuple values are written with parentheses and commas. When a tuple is
constructed, an appropriate tuple is generated on demand.

\hypertarget{vararg-tuple-types}{%
\subsubsection{Vararg Tuple Types}\label{vararg-tuple-types}}

The last parameter of a tuple type can be the special value
\texttt{Vararg}, which denotes number of trailing elements:

\begin{verbatim}
julia mytupletype = Tuple{AbstractString,Vararg{Int}}
Tuple{AbstractString, Vararg{Int64}}

julia> isa(("1",), mytupletype)
true

julia> isa(("1",1), mytupletype)
true

julia> isa(("1", 1, 2, 3.0), mytupletype)
false
\end{verbatim}

\hypertarget{named-tuple-types}{%
\subsubsection{Named Tuple Types}\label{named-tuple-types}}

Names tuples are instances of the \texttt{NamedTuple} type, which has
two parameters: a tuple of symbols giving the field names, and a tuple
giving the field types.

\begin{verbatim}
julia> typeof((a=1, b="hello"))
NamedTuple{(:a, :b), Tuple{Int64, String}}
\end{verbatim}

The \texttt{@NamedTuple} macro provides a more convenient struct-like
syntax for declaring \texttt{NamedTuple} types via \texttt{key::Type}
declarations, where an omitted \texttt{::Type} corresponds to
\texttt{Any}.

\begin{verbatim}
julia> @NamedTuple{a::Int, b::String}
NamedTuple{(:a, :b), Tuple{Int64, String}}

julia> @NamedTuple begin
         a::Int
         b::String
       end
NamedTuple{(:a, :b), Tuple{Int64, String}}
\end{verbatim}

A \texttt{NamedTuple} type can be used as a constructor, accepting a
single tuple argument. The constructed \texttt{NamedTuple} type can be
either a concrete type, with both parameters specified, or a type that
specifies only field names.

\hypertarget{parametric-primitive}{%
\subsubsection{Parametric Primitive}\label{parametric-primitive}}

Primitive types can also be declared parametrically. For example,
pointers are represented as primitive types which would be delcated in
\textbf{Julia} like:

\begin{verbatim}
# 32-bit system
primitive type Ptr{T} 32 end

# 64-bit system
primitive type Ptr{T} 64 end
\end{verbatim}

The slightly odd feature of these declarations as compared to typical
parametric composite types, is that the type parameter \texttt{T} is not
used in the definition of the type itself. It is just an abstract tag,
essentially defining an entire family of types with identical structure,
differentiated only by their type parameter. Thus
\texttt{Ptr\{Float64\}} and \texttt{Ptr\{Int64\}} are distinct types,
even though they have identical representations. And of course, all
specific types are subtypes of the umbrella \texttt{Ptr} type.

\begin{verbatim}
julia> Ptr{Float} <: Ptr
true

julia> Ptr{Int64} <: Ptr
true
\end{verbatim}

\hypertarget{unionall-types}{%
\subsection{UnionAll Types}\label{unionall-types}}

We have said that a parametric type like \texttt{Ptr} acts as a
supertype of all its instances (\texttt{Ptr\{Int64\}} etc.). How does
this work? \texttt{Ptr} itself cannot be a normal data type, since
without knowing the type of the referenced data the type clearly cannot
be used for memory operations. The answer is that \texttt{Ptr} (or other
parametric types like \texttt{Array}) is a different kind of type called
a \texttt{UnionAll} type. Such a type expresses the iterated union of
types for all values of some parameter.

\texttt{UnionAll} types are usually written using the keyword
\texttt{where}. For example \texttt{Ptr} could be more accurately
written as \texttt{Ptr\{T\}\ where\ T}, meaning all values whose type is
\texttt{Ptr\{T\}} for some value of \texttt{T}. In this context, the
parameter \texttt{T} is also often called a ``type variable'' since it
is like a variable that ranges over types. Each \texttt{where}
introduces a single type variable, so these expressions are nested for
types with multiple parameters, for example
\texttt{Array\{T,N\}\ where\ N\ where\ T}.

The type application syntax \texttt{A\{B,C\}} requires \texttt{A} to be
a \texttt{UnionAll} type, and first substitutes \texttt{B} for the
outermost type variable in \texttt{A}. The result is expected to be
another \texttt{UnionAll} type, into which \texttt{C} is then
substituted. So \texttt{A\{B,C\}} is equivalent to \texttt{A\{B\}\{C\}}.
This explains why it is possible to partially instantiate a type, as in
\texttt{Array\{Float64\}}: the first parameter value has been fixed, but
the second still ranges over all possible values. Using explicit
\texttt{where} syntax, any subset of parameters can be fixed. For
example, the type of all 1-dimensional arrays can be written as
\texttt{Array\{T,1\}\ where\ T}.

Type variables can be restricted with subtype relations.
\texttt{Array\{T\}\ where\ T\ \textless{}:\ Integer} refers to all
arrays whose element type is some kind of \texttt{Integer}. The syntax
\texttt{Array\{\textless{}:Integer\}} is a convenient shorthand for
\texttt{Array\{T\}\ where\ T\textless{}:Integer}. Type variables can
have both lower and upper bounds.
\texttt{Array\{T\}\ where\ Array\{T\}\ where\ T\textless{}:Integer}.
Type variables can have both locaer and upper bounds.
\texttt{Array\{T\}\ where\ Int\textless{}:T\textless{}:Number} refers to
all arrays of \texttt{Number}s that are able to contain \texttt{Int}s
(since \texttt{T} must be at least as big as \texttt{Int}). The syntax
\texttt{where\ T\textgreater{}:Int} also works to specify only the lower
bound of a type variable, and \texttt{Array\{\textgreater{}:Int\}} is
equivalent to \texttt{Array\{T\}\ where\ T\textgreater{}:Int}.

\hypertarget{singleton-types}{%
\subsection{Singleton Types}\label{singleton-types}}

Immutable composite types with no fields are called singletons.
Formally, if

\begin{enumerate}
\def\labelenumi{\arabic{enumi}.}
\tightlist
\item
  \texttt{T} is an immutable composite type (i.e.~defined with
  \texttt{struct}),
\item
  \texttt{a\ isa\ T\ \&\&\ b\ isa\ T} implies \texttt{a\ ===\ b}
\end{enumerate}

then \texttt{T} is a singleton type. \texttt{Base.issingletontype} can
be used to check if a type is a singleton type. Abstract types cannot be
singleton types by construction. From the definition, it follows that
there can be only one instance of such types:

\begin{verbatim}
julia> struct NoFields
       end

julia> NoFields() == NoFields()
true

julia> Base.issingletontype(NoFields)
true
\end{verbatim}

The \texttt{===} function confirms that the constructed instances of
\texttt{NoFields} are actually one and the same. Parametric types can be
singleton types when the above condition holds. For example,

\begin{verbatim}
julia> struct NoFieldsParam{T}
       end

julia> Base.issingletontype(NoFieldsParam) # Can't be a singleton type
false
\end{verbatim}

\hypertarget{types-of-functions}{%
\subsection{Types of functions}\label{types-of-functions}}

Each function has its own type, which is a subtype of \texttt{Function}.
Types of functions defined at top-level are singletons. When necessary,
you can compare them \texttt{===}. Closures also have their own type,
which is usually printed with names that end in
\texttt{\#\textless{}number\textgreater{}}. Names and types for
functions defined at different locations are distinct, but not
guaranteed to be printed the same way across sessions.

\hypertarget{typet-type-selectors}{%
\subsection{\texorpdfstring{\texttt{Type\{T\}} type
selectors}{Type\{T\} type selectors}}\label{typet-type-selectors}}

For each type \texttt{T}, \texttt{Type\{T\}} is an abstract parametric
type whose only instance is the object \texttt{T}. While \texttt{Type}
is part of \textbf{Julia}'s type hierarchy like any other abstract
parametric type, it is not commonly used outside method signatures
except in some special cases. Another important use case for
\texttt{Type} is sharpening field types which would otherwise be
captured less precisely, e.g.~as \texttt{DataType} in the example below
where the default constructor could lead to performance problems in code
relying on the precise wrapped type (similarly to abstract type
parameters).

\begin{verbatim}
julia> struct WrapType{T}
       value::T
       end

julia> WrapType(Float64) # default constructor, note DataType
WrapType{DataType}(Float64)

julia> WrapType(::Type{T}) where T = WrapType{Type{T}}(T)
WrapType

julia> WrapType(Float64) # sharpened constructor, note more precise Type{Float64}
WrapType{Type{Float64}}(Float64)
\end{verbatim}

\hypertarget{type-aliases}{%
\subsection{Type Aliases}\label{type-aliases}}

Sometimes it is convenient to introduce a new name for an aready
expressible type. This can be done with a simple assignment statement.
For example, \texttt{UInt} is aliased to either \texttt{UInt32} or
\texttt{UInt64} as is appropriate for the size of pointers on the
system:

\begin{verbatim}
# 32-bit system
julia> UInt
UInt32

# 64-bit system
julia> UInt
UInt64
\end{verbatim}

\hypertarget{operations-on-types}{%
\subsection{Operations on Types}\label{operations-on-types}}

Since types in \textbf{Julia} are themselves objects, ordinary functions
can operate on them. Some functions that are particularly useful for
working with or exploring types have already been introduced, such as
the \texttt{\textless{}:} operator, which indicates whether its left
hand operand is a subtype of its right hand operand. The \texttt{isa}
function tests if an object is of a given type and returns \texttt{true}
or \texttt{false}:

\begin{verbatim}
julia> isa(1, Int)
true

julia> isa(1, AbstractFloat)
false
\end{verbatim}

The \texttt{typeof} function, already used throughout the manual in
examples, return the type of its argument. Since types are objects, they
also have types: we can ask what their types are.

\hypertarget{custom-pretty-printing}{%
\subsection{Custom pretty-printing}\label{custom-pretty-printing}}

Often, one wants to customize how instances of a type are displayed.
This is accomplished by overloading the \texttt{show} function. For
example, suppose we define a type to represent complex numbers in polar
form:

\begin{verbatim}
julia> struct Polar{T<:Real} <: Number
         r::T
         t::T
        end

julia> Polar(r::Real, t::Real) = Polar(promote(r,t)...)
Polar
\end{verbatim}

\hypertarget{value-types}{%
\subsection{``Value types''}\label{value-types}}

In \textbf{Julia}, In Julia, you can't dispatch on a value such as
\texttt{true} or \texttt{false}. However, you can dispatch on parametric
types, and \texttt{Julia} allows you to include ``plain bits'' values
(Types, Symbols, Integers, floating-point numbers, tuples, etc.) as type
parameters. A common example is the dimensionality parameter in
\texttt{Array\{T,N\}}, where \texttt{T} is a type (e.g., Float64) but
\texttt{N} is just an \texttt{Int}.

You can create your own custom types that take values as parameters, and
use them to control dispatch of custom types. By way of illustration of
this idea, let's introduce the parametric type \texttt{Val\{x\}}, and
its constructor \texttt{Val(x)\ =\ Val\{x\}()}, which serves as a
customary way to exploit this technique for cases where you don't need a
more elaborate hierarchy.

\newpage

\hypertarget{methods}{%
\section{Methods}\label{methods}}

Recall from \emph{Functions} that a function is an object that maps a
tuple of arguments to a return value, or throws an exception if no
appropriate value can be returned. It is common for the same conceptual
function or operation to be implemented quite differently for different
types of arguments: adding two integers is very different from adding
two floating-point numbers, both of which are distinct from adding an
integer to a floating-point number.

To facilitate using many different implementations of the same concept
smoothly, functions need not be defined all at once, but can rather be
defined piecewise by providing specific behaviors for certain
combinations of argument types and counts. A definition of one possible
behavior for a function is called a \textbf{method}. The signatures of
method definitions can be annotated to indicate the types of arguments
in addition to their number, and more than a single method definition
may be provided. When a function is applied to a particular tuple of
arguments, the most specific method applicable to those arguments is
applied. Thus, the overall behavior of a function is a patchwork of the
behaviors of its various method definitions. If the patchwork is well
designed, even though the implementations of the methods may be quite
different, the outward behavior of the function will appear seamless and
consistent.

The choice of which method to execute when a function is applied is
called \emph{dispatch}. \textbf{Julia} allows the dispatch process to
choose which of a function's methods to call based on the number of
arguments given, and on the types of all of the function's arguments.
This is different than traditional object-oriented languages, where
dispatch occurs based only on the first argument, which often has a
special argument syntax, and is sometimes implied rather than explicitly
written as an argument. Using all of a function's arguments to choose
which method should be invoked, rather than just the first, is known as
\emph{multiple dispatch}. Multiple dispatch is particularly useful for
mathematical code, where it makes little sense to artificially deem the
operations to ``belong'' to one argument more than any of the others.

\hypertarget{defining-methods}{%
\subsection{Defining Methods}\label{defining-methods}}

Until now, we have, in our examples, defined only functions with a
single method having unconstrained argument types. Such functions behave
just like they would in traditional dynamically typed languages.
Nevertheless, we have used multiple dispatch and methods almost
continually without being aware of it: all of \textbf{Julia}'s standard
functions and operators have many methods defining their behavior over
various possible combinations of argument type and count.

When defining a function, on can optionally constrain the types of
parameters it is applicable to, using the \texttt{::} type-assertion
operator, introduced in the section on \texttt{Composite\ Types}.

\begin{verbatim}
julia> f(x::Float64, y::Float64) = 2x + y
f (generic function with 1 method)
\end{verbatim}

This function definition applies only to calls where \texttt{x} and
\texttt{y} are both values of type \texttt{Float64} so that applying it
to any other types of arguments will result in a \texttt{MethodError}.

To define a function with multiple methods, one simply defines the
function multiple times, with different numbers and types of arguments.
The first method definition for a function creates the function object,
and subsequent method definitions add new methods to the existing
function object. The most specific method definition matching the number
and types of the arguments will be executed when the function is
applied.

To find out what the signatures of those methods are, use the
\texttt{methods} function.

In the absence of a type declaration with \texttt{::}, the type of a
method parameter is \texttt{Any} by default, meaning that it is
unconstrained since all values in \textbf{Julia} are instances of the
abstract type \texttt{Any}.

\hypertarget{method-specializations.}{%
\subsection{Method specializations.}\label{method-specializations.}}

When you create multiple methods of the same function, this is sometimes
called \emph{specialization}. In this case, you're specializing the
function by adding additional methods to it: each new method is a new
specialization of the function.

There's another kind of specialization that occurs without programmer
intervention: \textbf{Julia}'s compiler can automatically specialize the
method for the specific argument types used. Such specializations are
not listed by \texttt{methods}, as this doesn't create new
\texttt{Methods}, but tools like \texttt{@code\_typed} allow you to
inspect such specialization.

\hypertarget{method-ambguities}{%
\subsection{Method Ambguities}\label{method-ambguities}}

It is possible to define a set of function methods such that there is no
unique most specific method applicable to some combinations of
arguments.

\begin{verbatim}
julia> g(x::Float64, y) = 2x + y
g (generic function with 1 method)

julia> g(x, y::Float64) = x + 2y
g (generic function with 2 methods)

julia> g(2.0, 3)
7.0

julia> g(2, 3.0)
8.0

julia> g(2.0, 3.0)
ERROR: MethodError: g(::Float64, ::Float64) is ambiguous.

Candidates:
  g(x::Float64, y)
    @ Main REPL[1]:1
  g(x, y::Float64)
    @ Main REPL[2]:1

Possible fix, define
  g(::Float64, ::Float64)

Stacktrace:
 [1] top-level scope
   @ REPL[3]:1
\end{verbatim}

Here the call \texttt{g(2.0,\ 3.0)} could be handled by either the
\texttt{g(Float64,\ Any)} or the \texttt{g(Any,\ Float64)} method, and
neither is more specific than the other. In such cases, \textbf{Julia}
raises a \texttt{MethodError} rather than arbitrarily picking a method.
You can avoid method ambiguities by specifying an appropriate method for
the intersection case.

\hypertarget{parametric-methods}{%
\subsection{Parametric Methods}\label{parametric-methods}}

Methods definitions can optionally have type parameters qualifying the
signature:

\begin{verbatim}
julia> same_type(x::T, y::T) where {T} = true
same_type (generic function with 1 method)

julia> same_type(x,y) = false
same_typ (generic function with 2 methods)
\end{verbatim}

The first method applies whenever both arguments are of the same
concrete type, regardless of what type that is, while the second method
acts as a catch-all, coveringj all other cases. Thus, overall, this
defines a boolean function that checks whether its two arguments are of
the same type:

\begin{verbatim}
julia> same_type(1, 2)
true

julia> same_type(1, 2.0)
false

julia> same_type(Int32(1), Int64(2))
false
\end{verbatim}

\hypertarget{redefining-methods}{%
\subsection{Redefining Methods}\label{redefining-methods}}

When redefining a method or adding new methods, it is important to
realize that these changes don't take effect immediately. This is key to
\textbf{Julia}'s ability to statically infer and compile code to run
fast, without the usual \emph{JIT} tricks and overhead. Indeed, any new
method definition won't be visible to the current runtime environment,
including \texttt{Tasks} and \texttt{Threads} (and any previously
defined \texttt{@generated} functions). Let's start by an example to see
what this means:

\begin{verbatim}
julia> function tryeval()
         @eval newfun() = 1
         newfun()
       end
tryeval (generic function with 1 method)

julia> tryeval()
ERROR: MethodError: no method matching newfun()
The applicable method may be too new: running in world age xxxx6, while current world is xxxx7.

Closest candidates are:
  newfun() (method too new to be called from this world context.)
   @ Main REPL[4]:2

Stacktrace:
 [1] tryeval()
   @ Main ./REPL[4]:3
 [2] top-level scope
   @ REPL[5]:1

julia> newfun()
1
\end{verbatim}

In this example, observe that the new definition for \texttt{newfun} has
been created, but can't be immediately called. The new global is
immediately visible to the tryeval function, so you could write
\texttt{return\ newfun} (without parentheses). But neither you, nor any
of your callers, nor the functions they call, or etc. can call this new
method definition!

But there's an exception: future calls to \texttt{newfun} from the
\texttt{REPL} work as expected, being able to both see and call the new
definition of \texttt{newfun}.

However, future calls to \texttt{tryeval} will continue to see the
definition of \texttt{newfun} as it was at the previous statement at the
\texttt{REPL}, and thus before that call to \texttt{tryeval}.

The implementation of this behavior is a ``world age counter''. This
monotonically increasing value tracks each method definition operation.
This allows describing ``the set of method definitions visible to a
given runtime environment'' as a single number, or ``world age''. It
also allows comparing the methods available in two worlds just by
comparing their ordinal value. In the example above, we see that the
``current world'' (in which the method \texttt{newfun} exists), is one
greater than the task-local ``runtime world'' that was fixed when the
execution of \texttt{tryeval} started.

Sometimes it is necessary to get around that behavior; to do so one will
call the function using \texttt{Base.invokelatest}:

\begin{verbatim}
julia> function tryeval2()
         @eval newfun2() = 2
         Base.invokelatest(newfun2)
       end
tryeval2 (generic function with 1 method)

julia> tryeval2()
2
\end{verbatim}

Finally, let's take a look at some more complex examples where this rule
comes into play. Define a function \texttt{f} which initially has one
method:

\begin{verbatim}
julia> f(x) = "original definition"
f (generic function with 1 method)
\end{verbatim}

Start some other operations that use \texttt{f}:

\begin{verbatim}
julia> g(x) = f(x)
g (generic function with 1 method)

julia> t = @async f(wait()); yield();
\end{verbatim}

Now we add some new methods to \texttt{f}:

\begin{verbatim}
julia> f(x::Int) = "definition for Int"
f (generic function with 2 methods)

julia> f(x::Type{Int}) = "definition for Type{Int}"
f (generic function with 3 methods)
\end{verbatim}

Compare how these results differ:

\begin{verbatim}
julia> f(1)
"defintion for Int"

julia> g(1)
"definition for Int"

julia> t = @async f(waith()); yield();

julia> fetch(schedule(t, 1))
"definition for Int"
\end{verbatim}

\hypertarget{function-like-objects}{%
\subsection{Function-like objects}\label{function-like-objects}}

Methods are associated with types, so it is possible to make any
arbitrary \textbf{Julia} object ``callable'' by adding methods to its
type (Such ``callable'' objects are sometimes called ``functors''). For
example, you can define a type that stores the coefficients of a
polynomial, but behaves like a function evaluating the polynomial:

\begin{verbatim}
julia> struct Polynomial{R}
         coeffs::Vertor{R}
       end

julia> function(p::Polynomial)(x)
         v = p.coeffs[end]
         for i = (length(p.coeffs)-1):-1:1
           v = v * x + p.coeffs[i]
         end
         return v
       end
\end{verbatim}

\newpage

\hypertarget{constructors}{%
\section{Constructors}\label{constructors}}

\emph{Constructors} are functions that create new objects (specifically,
instances of Composite Types). In \textbf{Julia}, type objects also
serve as constructor functions: they create new instances of themselves
when applied to an argument tuple as a function. This much was already
mentioned briefly when composite types were introduced. For example:

\begin{verbatim}
julia> struct Foo
         bar
         baz
       end

julia> foo = Foo(1, 2)
Foo(1, 2)

julia> foo.bar
1

julia> foo.baz
2
\end{verbatim}

For many types, forming new objects by binding their field values
together is all that is ever needed to create instances. However, in
some cases more functionality is required when creating composite
objects. Sometimes invariants must be enforced, either by checking
arguments or by transforming them. \emph{Recursive data structures},
especially those that may be self-referential, often cannot be
constructed cleanly without first being created in an incomplete state
and then altered programmatically to be made whole, as separate step
from object creation. Sometimes, it's just convenient to be able to
construct objects with fewer or different types of parameters than they
have fields. \textbf{Julia}'s system for object construction addresses
all of these cases and more.

\hypertarget{outer-constructor-methods}{%
\subsection{Outer Constructor Methods}\label{outer-constructor-methods}}

A constructor is just like any other function in Julia in that its
overall behavior is defined by the combined behavior of its methods.
Accordingly, you can add functionality to a constructor by simply
defining new methods.

\hypertarget{inner-constructor-methos}{%
\subsection{Inner Constructor Methos}\label{inner-constructor-methos}}

While outer constructor methods succeed in addressing the problem of
providing additional convenience methods for constructing objects, they
fail to address the other two use cases i.e.~\emph{enforcing invariants}
and \emph{allowing construction of self-referential objects}. For these
problems, one needs inner constructor methods. An inner constructor
method is like an outer constructor method, except for two differences:

\begin{enumerate}
\def\labelenumi{\arabic{enumi}.}
\tightlist
\item
  It is declared inside the block of a type declaration, rather than
  outside of it like normal methods.
\item
  It has access to a special locally existent function called
  \texttt{new} that creates objects of the block's type.
\end{enumerate}

For example, suppose one wants to declare a type that holds a pair of
real numbers, subject to the constraint that the first number is not
greater than the second one:

\begin{verbatim}
julia> struct OrderedPair
         x::Real
         y::Real
         OrderedPair(x,y) = x > y ? error("out of order") : new(x, y)
       end

julia> OrderedPair(3.75, 3.76)
OrderedPair(3.75, 3.76)

julia> OrderedPair(-0.706, -0.707)
ERROR: out of order
Stacktrace:
 [1] error(s::String)
   @ Base ./error.jl:35
 [2] OrderedPair(x::Float64, y::Float64)
   @ Main ./REPL[12]:4
 [3] top-level scope
   @ REPL[14]:1
\end{verbatim}

If the type were declared mutable, you could reach in and directly
change the field values to violate this invariant. Of course, messing
around with an object's internals uninvited is bad practice. You (or
someone else) can also provide additional outer constructor methods at
any later point, but once a type is declared, there is no way to add
more inner constructor methods. Since outer constructor methods can only
create objects by calling other constructor methods, ultimately, some
inner constructor must be called to create an object. This guarantees
that all objects of the declared type must come into existence by a call
to one of the inner constructor methods provided with the type, thereby
giving some degree of enforcement of a type's invariants.

If any inner constructor method is defined, no default constructor
method is provided: it is presumed that you have supplied yourself with
all the inner constructors you need. The default constructor is
equivalent to writing your own inner constructor method that takes all
of the object's fields as parameters (constrained to be of the correct
type, if the corresponding field has a type), and passes them to new,
returning the resulting object:

\begin{verbatim}
julia> struct Foo
         bar
         baz
         Foo(bar, baz) = new(bar, baz)
       end
\end{verbatim}

\hypertarget{incomplete-initialization}{%
\subsection{Incomplete Initialization}\label{incomplete-initialization}}

The final problem which has still not been addressed is construction of
self-referential objects, or more generally, recursive data structures.
Since the fundamental difficulty may not be immediately obvious, let us
briefly explain it. Consider the following recursive type declaration:

\begin{verbatim}
julia> mutable struct SelfReferential
         obj::SelfReferential
       end
\end{verbatim}

This type may appear innocuous enough, until one considers how to
construct an instance of it. If a is an instance of SelfReferential,
then a second instance can be created by the call:

\begin{verbatim}
julia> b = SelfReferential(a)
\end{verbatim}

But how does one construct the first instance when no instance exists to
provide as a valid value for its obj field? The only solution is to
allow creating an incompletely initialized instance of SelfReferential
with an unassigned obj field, and using that incomplete instance as a
valid value for the obj field of another instance, such as, for example,
itself.

To allow for the creation of incompletely initialized objects,
\textbf{Julia} allows the new function to be called with fewer than the
number of fields that the type has, returning an object with the
unspecified fields uninitialized. The inner constructor method can then
use the incomplete object, finishing its initialization before returning
it. Here, for example, is another attempt at defining the
\texttt{SelfReferential} type, this time using a zero-argument inner
constructor returning instances having \texttt{obj} fields pointing to
themselves:

\begin{verbatim}
julia> mutable struct SelfReferential
         obj::SelfReferential
         SelfReferential() = (x = new(); x.obj = x)
       end
\end{verbatim}

\hypertarget{parametric-constructors}{%
\subsection{Parametric Constructors}\label{parametric-constructors}}

Parametric types add a few wrinkles to the constructor story. Recall
from Parametric Types that, by default, instances of parametric
composite types can be constructed either with explicitly given type
parameters or with type parameters implied by the types of the arguments
given to the constructor.

\end{document}
